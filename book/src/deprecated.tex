% !TEX root = ./lshort.tex
\chapter{Things You Shouldn't Use}
\bgroup
\ExplSyntaxOn
\NewDocumentCommand{\instead}{mm}{
  \begin{trivlist}
    \item
    \begin{tabular}{@{}l@{}l@{}}
      \toprule
      Instead~of                         & Use \\
      \midrule
      \begin{minipage}[t]{0.5\textwidth}
        #1
      \end{minipage} &
      \begin{minipage}[t]{0.5\textwidth}
        #2
      \end{minipage}        \\
      \bottomrule
    \end{tabular}
  \end{trivlist}
}

\cs_set:Nn \__lshort_fix_verb_arg:n {
  \tl_set:Nn \l_tmpa_tl {#1}
  \exp_args:NNx\tl_replace_all:Nnn \l_tmpa_tl {\char_generate:nn{13}{12}} {\\}
  \tl_replace_all:Nnn \l_tmpa_tl {~} {\mbox{~}}
  \tl_use:N \l_tmpa_tl
}

\NewDocumentCommand{\chto}{+v+v}{
  \begin{trivlist}
    \item
    \begin{minipage}{0.4\textwidth}
      \ttfamily
      \__lshort_fix_verb_arg:n {#1}
    \end{minipage}
    \hspace{\stretch{1}}
    \(\longrightarrow\)
    \hspace{\stretch{1}}
    \begin{minipage}{0.4\textwidth}
      \ttfamily
      \__lshort_fix_verb_arg:n {#2}
    \end{minipage}
  \end{trivlist}
}
\NewDocumentCommand{\vchto}{+v+v}{
  \begin{center}
    \ttfamily
    \__lshort_fix_verb_arg:n {#1}\\[.2cm]
    \(\big\downarrow\)\\[.2cm]
    \__lshort_fix_verb_arg:n {#2}
  \end{center}
}

\NewCommandCopy{\oldsec}{\section}
\RenewDocumentCommand{\section}{m}{\oldsec{\ldots{}~for~#1}}
\ExplSyntaxOff

\section{Display Math}
\instead{
\ai{\&\&}
} {
\ci{[}, \ci{]} \\
\ei*{equation*}\\
\ei*{displaymath}\\
(All with the \pai{amsmath} package.)
}

\ai{\&\&} is \hologo{plainTeX} syntax and it cannot be modified. The
\ci{displaymath} and \ci{[} are \hologo{LaTeX} commands that are a bit better
in terms of spacing and features and can be redefined. With \pai{amsmath} both
of them are redefined to be synonyms for the \ei{equation*} which produces
optimal and consistent spacing.

\chto|$$ 2 + 2 = 4 $$||\[ 2 + 2 = 4 \]| %chktex 45

\section{Typesetting Math}
\instead{
  \ci{over} \\
  \ci{choose}
} {
  \ci{frac} \\
  \ci{binom} (\pai{amsmath})
}

These are \hologo{plainTeX} commands for producing fractions and binomial
coefficients. Due to their unusual syntax, their use may lead to ambiguous
code.
\begin{chktexignore}
\chto|{a \over b}||\frac{a}{b}|
\chto|{a \choose b}|
|\usepackage{amsmath}
% ...
\binom{a}{b}|
\end{chktexignore}

\section{Defining New Commands}\label{sec:def}
\instead{
  \ci{newcommand} \\
  \ci{renewcommand} \\
  \ci{def}
} {
  \ci{NewDocumentCommand} \\
  \ci{RenewDocumentCommand} \\
  \ci{DeclareDocumentCommand}
}

Both \ci{newcommand} and \ci{renewcommand} are \hologo{LaTeX2e} macros. They
are not as expressive as \texttt{...DocumentCommand} family  %chktex 11
described in Section~\ref{sec:new_commands}. Their syntax only allows a single
optional argument with default value (specified as second optional argument)
followed by few mandatory ones (specified as number in first optional
argument).
\begin{chktexignore}
  \vchto|\newcommand{\foo}[4][bar]{ ... }|
  |\NewDocumentCommand{\foo}{O{bar}mmmm}{ ... }|
\end{chktexignore}

The \ci{def} command is a \hologo{plainTeX} primitive. It \emph{always} defines
the command, even if it was already defined. This is usually not desirable, but
if it's needed you can use \ci{DeclareDocumentCommand}.
\begin{chktexignore}
  \vchto|\def\foo#1#2#3{ ... }|
  |\NewDocumentCommand{\foo}{mmm}{ ... }|
\end{chktexignore}

\section{Copying Commands}
\instead{
  \ci{let}
} {
  \ci{NewCommandCopy} \\
  \ci{RenewCommandCopy} \\
  \ci{DeclareCommandCopy}
}

Similarly to~\ref{sec:def}, \ci{let} is a \hologo{plainTeX} primitive that does
does not guard against accidental redefinition. Moreover it does not work
correctly with some \hologo{LaTeX} commands. Use \ci{NewCommandCopy} as
described in Section~\ref{sec:copyingcommands}.
\chto|\let\foo\bar||\NewCommandCopy\foo\bar|

\section{Aligning Equations}
\instead{
  \ei*{eqnarray} \\
  \ei*{eqnarray*}
}{
  \ei*{align}  \\
  \ei*{align*} \\
  (Both in \pai{amsmath}.)
}

\ei{eqnarray} and \ei{eqnarray*} are \hologo{LaTeX} environments that allow
aligning equations. However spacing around the binary operators in these is far
from ideal. Therefore it is recommended to always use \ei{align} environments
from \pai{amsmath} or \ei{IEEEeqnarray} as described in
Section~\ref{sec:IEEEeqnarray}.
\begin{chktexignore}
\chto
|\begin{eqnarray}
  f(x) & = &  1 + 2 \\
  g(x) & > & 52
\end{eqnarray}|
|\usepackage{amsmath}
% ...
\begin{align}
  f(x) & = 1 + 2 \\
  g(x) & > 52
\end{align}|
\end{chktexignore}

\section{Changing Fonts}
\instead{
  \ci{bf} \\
  \ci{rm} \\
  \ci{sf} \\
  \ci{tt} \\
  \ci{it} \\
  \ci{sc} \\
  \ci{sl}
}{
  \ci{bfseries} \\
  \ci{rmfamily} \\
  \ci{sffamily} \\
  \ci{ttfamily} \\
  \ci{itshape}  \\
  \ci{scshape}  \\
  \ci{slshape}
}

These are \hologo{plainTeX} commands. Each of them resets the font to normal
before changing it, so for example bold italics cannot be achieved using them.
Use newer commands as described in Section~\ref{sec:fontsize}.
\chto|{\bf foo}||{\bfseries foo}|

\section{Changing Text Alignment}
\instead{
  \ci{flushleft}  \\
  \ci{flushright} \\
  \ci{center}
}{
  \ci{FlushLeft} \\
  \ci{FlushRight} \\
  \ci{Center} \\
  (\pai{ragged2e})
}

The default \hologo{LaTeX2e} commands for changing text alignment make it
nearly impossible to hyphenate words inside them. Therefore it is recommended
to use the \pai{ragged2e} equivalents, as described in
Section~\ref{sec:ragged}, that make the text less \enquote*{ragged} than it
should be.
\begin{chktexignore}  
\chto|\begin{center}
  text
\end{center}|
|\usepackage{ragged2e}
% ...
\begin{Center}
  text
\end{Center}|
\end{chktexignore}

\section{Typesetting Quotations}
\instead{
  \texttt{``} \\
  \texttt{,,} \\
  \texttt{<<} \\
  \texttt{>>} \\
  \texttt{''}
}{
  \ci{enquote} \\
  \ci{enquote*}
}

The \TeX{} version of entering quotes was to rely on ligatures for the given
quotation mark. This method is not context aware and cannot be customized.
Using \pai{csquotes} package as described in Section~\ref{sec:csquotes} allows
greater control over the typesetting of the quotations.

\chto|``quote''||\usepackage{csquotes}
% ...
\enquote{quote}|

\egroup
