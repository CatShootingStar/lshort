% !TEX root = ./lshort.tex
%%%%%%%%%%%%%%%%%%%%%%%%%%%%%%%%%%%%%%%%%%%%%%%%%%%%%%%%%%%%%%%%%
% Contents: Specialities of the LaTeX system
% $Id$
%%%%%%%%%%%%%%%%%%%%%%%%%%%%%%%%%%%%%%%%%%%%%%%%%%%%%%%%%%%%%%%%%

\chapter{Specialities}
\begin{intro}
  When putting together a large document, \LaTeX{} will help with some special
  features like index generation, automatic linking to relevant pages and other
  things. A much more complete description of specialities and enhancements
  possible with \LaTeX{} can be found in the {\normalfont\manual{}} and
  {\normalfont \companion}.
\end{intro}

\section{Indexing} \label{sec:indexing}
A very useful feature of many books is their \wi{index}. With \LaTeX{}
and the support program \texttt{makeindex},\footnote{On systems not
  necessarily supporting
  filenames longer than 8~characters, the program may be called
  \texttt{makeidx}.} an index can be generated quite easily.  This
introduction will only explain the basic index generation commands.
For a more in-depth view, please refer to \companion.  \index{makeindex
  program} \index{makeidx package}

To enable their indexing feature of \LaTeX{}, the \pai{makeidx} package
must be loaded in the preamble with
\begin{lscommand}
\verb|\usepackage{makeidx}|
\end{lscommand}
\noindent and the special indexing commands must be enabled by putting
the
\begin{lscommand}
  \ci{makeindex}
\end{lscommand}
\noindent command in the preamble.

The content of the index is specified with
\begin{lscommand}
  \ci{index}\verb|{|\emph{key}@\emph{formatted\_entry}\verb|}|
\end{lscommand}
\noindent commands, where \emph{formatted\_entry} will appear in the index
and \emph{key} will be used for sorting.  The \emph{formatted\_entry} is
optional. If it is missing the \emph{key} will be used. You enter the index
commands at the points in the text that you want the final index entries to
point to.  Table~\ref{index} explains the syntax with several examples.

\begin{table}[!tp]
  \centering
\caption{Index Key Syntax Examples.}
\label{index}
\begin{tabular}{@{}lll@{}}
  \toprule
  Code &Index Entry &Comment\\
  \midrule
  \rule{0pt}{1.05em}\verb|\index{hello}| &hello, 1 &Plain entry\\
  \verb|\index{hello!Peter}|   &\hspace*{2ex}Peter, 3 &Subentry under `hello'\\
  \verb|\index{Sam@\emph{Sam}}|     &\emph{Sam}, 2& Formatted entry\\
  \verb|\index{Kaese@\emph{K\"ase}}|     &\emph{K\"ase}, 33& Formatted entry\\
  \verb.\index{ecole@\'ecole}.     &\'ecole, 4& Formatted entry\\
  \verb.\index{Jenny|emph}.     &Jenny, \emph{3}& Formatted page number\\
  \verb.\index{Joe@\emph{Joe}|emph}.     &\emph{Joe}, \emph{5}& Formatted page number\\
\bottomrule
\end{tabular}
\end{table}

When the input file is processed with \LaTeX{}, each \verb|\index|
command writes an appropriate index entry, together with the current
page number, to a special file. The file has the same name as the
\LaTeX{} input file, but a different extension (\verb|.idx|). This
\eei{.idx} file can then be processed with the \texttt{makeindex}
program:

\begin{lscommand}
  \texttt{makeindex} \emph{filename}
\end{lscommand}
The \texttt{makeindex} program generates a sorted index with the same base
file name, but this time with the extension \eei{.ind}. If now the
\LaTeX{} input file is processed again, this sorted index gets
included into the document at the point where \LaTeX{} finds
\begin{lscommand}
  \ci{printindex}
\end{lscommand}

The \pai{showidx} package that comes with \LaTeXe{} prints out all
index entries in the left margin of the text. This is quite useful for
proofreading a document and verifying the index.

Note that the \ci{index} command can affect your layout if not used carefully.

\begin{example}
My Word \index{Word}. As opposed
to Word\index{Word}. Note the
position of the full stop.
\end{example}

\texttt{makeindex} has no clue about characters outside the ASCII range. To
get the sorting correct, use the \verb|@| character as shown in the K\"ase
and \'ecole examples above.

\section{Fancy Headers}
\label{sec:fancy}

\subsection{Basic commands}

The \pai*{fancyhdr} package provides a few simple commands that allow you to
customise the header and footer lines of your document. It defines an
additional page style \cargv{fancy} that can be customised to our liking. By
default it only adds a line separating header from the page body.
\begin{example}[standalone, paperheight=3cm]
\geometry{includefoot, includehead, headsep=.5em, footskip=1em} %!hide 
%!showbegin %!hide
\documentclass{article}

%!showend %!hide
\usepackage{fancyhdr}
\pagestyle{fancy}

\begin{document}
\noindent %!hide
This statement is false.
\end{document}
\end{example}

The basic command of the package is
\begin{lscommand}
  \ci{fancyhf}[places][field]
\end{lscommand}
The \carg{places} is a comma separated list of places to put the \carg{field}
argument. There are total of 12 different places and each is identified by a
combination of three letters:
\begin{itemize}
  \item The first letter tells whether it is in a header (\cargv{H}) or in a
  footer (\cargv{F}).
  \item The second letter tells which part of the header\slash footer should be
  occupied. The possible positions are left (\cargv{L}), centre (\cargv{C}) and
  right (\cargv{R}).
  \item The third letter tells whether the field should be printed on even
  (\cargv{E}) or odd (\cargv{O}) pages. If the document is not two sided then
  all pages are treated as odd.
\end{itemize}
So for example the combination \cargv{FCE} identifies the centre part of the
footer on even pages. If any of the letters is omitted then the identifier
points toward all positions specifiable by the omitted letter, so for example
\cargv{HR} means right part of the header on both odd and even pages.
\begin{example}[template=empty, standalone, paperheight=2.3cm, paperwidth=6cm, to_page=2,vertical_pages]
\documentclass[twoside]{article}

%!hidebegin
\usepackage[paperheight=\height,
    paperwidth=\width,
    margin=0.3cm,
    includefoot,
    includehead,
    headsep=.5em,
    footskip=1em
]{geometry}
%!hideend
\usepackage{fancyhdr} 
\pagestyle{fancy}

\fancyhf[HCE]{A}
\fancyhf[L]{\emph{B}}
\fancyhf[FR]{\textbf{C}}
\fancyhf[HCO, HRE]{\textsl{D}}

\begin{document}
\noindent %!hide
The next statement is false.
The previous statement is true.
\end{document}
\end{example}

There are two additional commands \ci{fancyhead} and \ci{fancyfoot} that work
the same way except that they imply respectively \cargv{H} and \cargv{F} in
their \carg{places} argument if it's missing.
\begin{example}[standalone, paperheight=3.5cm]
\geometry{includefoot, includehead, headsep=.5em, footskip=1em} %!hide
\sloppy %!hide
\usepackage{csquotes} %!hide
\usepackage{fancyhdr}%!hide
\pagestyle{fancy}%!hide

\fancyhead[L]{A}
\fancyfoot[R]{B}

\begin{document}%!hide
\noindent %!hide
\enquote{Yields a falsehood when
appended to its own quotation}
yields a falsehood when appended
to its own quotation.
\end{document}%!hide
\end{example}

The lines drawn by the \pai{fancyhdr} package may also be customised. To change
their thickness redefine the
\begin{lscommand}
  \ci{headrulewidth} \\
  \ci{footrulewidth}
\end{lscommand}
macros to the desired length.
\begin{example}[standalone, paperheight=3cm]
\geometry{includefoot, includehead, headsep=.5em, footskip=1em} %!hide
\sloppy %!hide
\usepackage{fancyhdr} %!hide
\pagestyle{fancy} %!hide
\RenewDocumentCommand{\headrulewidth}{}{.2cm}
\RenewDocumentCommand{\footrulewidth}{}{.5cm}

\begin{document}%!hide
\noindent %!hide
Do not read this sentence.
\end{document}%!hide
\end{example}

By default headers and footers are as long as the text on the page. If you want
to extend\slash shorten them use the
\begin{lscommand}
  \ci{fancyhfoffset}[places][offset]
\end{lscommand}
command. The \carg{places} argument is the same as in \ci{fancyhf} except that
it cannot contain \cargv{C}.
\begin{example}[standalone, paperheight=3cm]
\geometry{includehead, includefoot, headsep=.5em, footskip=1em} %!hide
\sloppy %!hide
\usepackage{fancyhdr}%!hide
\pagestyle{fancy}%!hide
\fancyhfoffset[L]{-1cm}
\fancyhfoffset[R]{.2cm}

\begin{document}%!hide
\noindent %!hide
If this sentence is true,
then \(2 + 2 = 5\).
\end{document}%!hide
\end{example}
Similarly to \ci{fancyf}, \ci{fancyheadoffset} and \ci{fancyfootoffset} are
used the same way except they only modify header\slash footer.

\subsection{Contents of the headers}

The default footer of the article class contains the current page number. To
use it inside the fancy header simply use the command \ci{thepage}.
\begin{example}[standalone, paperheight=2.5cm, to_page=2, vertical_pages]
\geometry{includehead, includefoot, headsep=.5em, footskip=1em} %!hide
\sloppy %!hide
\usepackage{fancyhdr}%!hide
\pagestyle{fancy}%!hide
\fancyhf{Page~\thepage}

\begin{document}%!hide
\noindent %!hide
This statement is dedicated to
all statements that are not
dedicated to themselves. 
\end{document}%!hide
\end{example}

It is often useful to have the header and footer contain information based on
the contents of the page. These are called \enquote{marks} in \LaTeX{}
terminology. Before we talk about the default ones let's consider how you can
define your own using the \pai{extramarks}:\footnote{\pai{extramarks} is part
of \pai*{fancyhdr}.}
\begin{lscommand}
  \ci*{extramarks}[left:m ! right:m] \\
  \ci{firstleftxmark} \\
  \ci{firstrightxmark} \\
  \ci{lastleftxmark} \\
  \ci{lastrightxmark}
\end{lscommand}
The \ci{extramarks} is used to set the contents of \carg{left} and \carg{right}
marks. Then you can access those inside the headers by using the appropriate
command. The \texttt{first}- command refer to the first mark occurring on the
page, while the \texttt{last}- refer to the last one.
\begin{example}[standalone, paperheight=4cm]
\geometry{includehead, includefoot, headsep=.5em, footskip=1em} %!hide
\sloppy %!hide
\usepackage{fancyhdr}%!hide
\usepackage{extramarks}
\pagestyle{fancy}%!hide

\fancyhead[L]{\firstleftxmark}
\fancyhead[R]{\lastleftxmark}
\fancyfoot[L]{\firstrightxmark}
\fancyfoot[R]{\lastrightxmark}

\begin{document} %!hide
\noindent %!hide
The second statement is false.
\extramarks{One}{2 is false}
The third statement is false.
\extramarks{Two}{1 is false}
The first statement is false.
\extramarks{Three}{3 is false}
\end{document} %!hide
\end{example}

Let us now look at the default marks defined by \LaTeX{}. After loading
\pai{extramarks} these may be set and accessed similarly:
\begin{lscommand}
  \ci*{markboth}[left:m ! right:m] \\
  \ci{firstleftmark} \\
  \ci{firstrightmark} \\
  \ci{lastleftmark} \\
  \ci{lastrightmark}
\end{lscommand}
The only difference between these and the extra marks is that \LaTeX{} classes
automatically fill them, so if you are not careful you may lose your content.
For example in the article class, \carg{left} is set by the \ci{section}
command, while \carg{right} is set by the \ci{subsection} command.
\begin{example}[standalone, paperheight=5cm]
\geometry{includehead, includefoot, headsep=.5em, footskip=1em} %!hide
\sloppy %!hide
\usepackage{fancyhdr}%!hide
\usepackage{extramarks}%!hide
\pagestyle{fancy}%!hide
\fancyhead[L]{\firstleftmark}
\fancyhead[R]{\lastleftmark}
\fancyfoot[L]{\firstrightmark}
\fancyfoot[R]{\lastrightmark}

\begin{document}%!hide
\section{First}
\subsection{Sub}
\section{Second}
\subsection{Sub}
\end{document}%!hide
\end{example}
Note that \ci{firstrightmark} is empty in the above example. This is caused by
the fact that \ci{section} commands sets both marks, leaving the right one
empty.

You may have noticed that section titles are typeset in uppercase when given by
marks. To disable this use the \ci{nouppercase} command inside the \ci{fancyhf}
command.
\begin{example}[standalone, paperheight=3cm]
\geometry{includehead, includefoot, headsep=.5em, footskip=1em} %!hide
\sloppy %!hide
\usepackage{fancyhdr}%!hide
\usepackage{extramarks}%!hide
\pagestyle{fancy}%!hide
\fancyhead[R]{\nouppercase{\firstleftmark}}

\begin{document}%!hide
\section{Today}
is opposite day.
\end{document}%!hide
\end{example}

\subsection{Advanced commands}

If you want even more control over the section titles you may redefine the
\ci{sectionmark}.\footnote{Or \ci{chaptermark}, \ci{subsectionmark} \ldots} The
command receives the section title as its first argument, while the section
number is available as \ci{thesection}.
\begin{example}[standalone, paperheight=4cm]
\geometry{includehead, includefoot, headsep=.5em, footskip=1em} %!hide
\sloppy %!hide
\usepackage{fancyhdr}%!hide
\usepackage{extramarks}%!hide
\pagestyle{fancy}%!hide
\fancyhead[R]{\firstleftmark}
\RenewDocumentCommand{\sectionmark}{m}{%
  \markboth{Section no.\,\thesection: #1}{}%
}

\begin{document}%!hide
\section{Person}
There exists a person such that
if they are reading this then
everybody is reading this.
\end{document}%!hide
\end{example}
In case you want to change only the right mark, the \ci{markright} command is
available.

By default the field in the centre of header\slash footer will expand equally
to the left and to the right. This is usually the desired behaviour but in some
cases it may overlap with either left or right text, while still having some
space on the other side.
\begin{example}[standalone, paperheight=3cm]
\geometry{includehead, includefoot, headsep=.5em, footskip=1em} %!hide
\sloppy %!hide
\usepackage{fancyhdr}%!hide
\usepackage{extramarks}%!hide
\pagestyle{fancy}%!hide
\fancyhead[L]{\thepage}
\fancyhead[C]{\firstleftmark}
\fancyhead[R]{Jane Doe}

\begin{document}%!hide
\section{Section}
Section by Jane Doe
\end{document}%!hide
\end{example}
In this situation you may want to use the
\begin{lscommand}
  \ci*{fancycenter}[distance:o ! stretch:o ! left:m ! centre:m ! right:m]
\end{lscommand}
command that automatically shifts the centre toward the shorter text. The
\carg{distance} is the minimal distance that the elements are always surrounded
with (\cargv{1em} by default), while the \carg{stretch} controls the preference
for shifting the \carg{centre}: \cargv{1} means shift only when- and as much
as necessary, higher numbers will start shifting sooner and more aggressively,
default is \cargv{3}. This command writes over the whole header\slash footer
space so it should only be put in one place (typically \cargv{C}) and other
places (\cargv{L,R}) should be empty.
\begin{example}[standalone, paperheight=3cm]
\geometry{includehead, includefoot, headsep=.5em, footskip=1em} %!hide
\sloppy %!hide
\usepackage{fancyhdr}%!hide
\usepackage{extramarks}%!hide
\pagestyle{fancy}%!hide
\fancyhead[L,R]{}
\fancyhead[C]{%
  \fancycenter%
    {\thepage}%
    {\firstleftmark}%
    {Jane Doe}%
}

\begin{document}%!hide
\section{Section}
Section by Jane Doe
\end{document}%!hide
\end{example}

You may want to present different headers\slash footers when the corresponding
page starts with a float or ends with a footnote. The \pai{fancyhdr} package
defines four commands that let you achieve this.
\begin{lscommand}
  \ci*{iftopfloat}[true branch:m ! false branch:m] \\
  \ci*{ifbotfloat}[true branch:m ! false branch:m] \\
  \ci*{iffloatpage}[true branch:m ! false branch:m] \\
  \ci*{iffootnote}[true branch:m ! false branch:m]
\end{lscommand}
\ci{iftopfloat}/\ci{ifbotfloat} execute \carg{true branch} if a float sits at
the top\slash bottom of the page. Similarly \ci{iffloatpage} checks whether
the page is special float-only page and \ci{iffootnote} checks if footnote is
at the bottom of the page.
\begin{example}[standalone, paperheight=4cm, to_page=2, vertical_pages]
\geometry{includehead, includefoot, headsep=.5em, footskip=1em} %!hide
\sloppy %!hide
\usepackage{fancyhdr}%!hide
\usepackage{extramarks}%!hide
\pagestyle{fancy}%!hide

\fancyhead[C]{%
  \iftopfloat{%
    A float is below me%
  } {}
}
\fancyfoot[C]{%
  \iffootnote{%
    A footnote is above me%
  } {%
    \thepage%
  }%
}

\begin{document}%!hide
\noindent %!hide
Ignore this footnote.%
\footnote{Read this.}
\begin{figure}[t]
  \centering
  A floating float.
  \caption{Hmm}
\end{figure}
\end{document}%!hide
\end{example}

The lines in headers and footers are created by invoking \ci{headrule} and
\ci{footrule} commands. If you want finer control over the lines consider
redefining these macros. Use \ci{headruleskip} and \ci{footruleskip} to raise
or lower them if necessary.
\begin{example}[standalone, paperheight=4cm]
\geometry{includehead, includefoot, headsep=.5em, footskip=1em} %!hide
\sloppy %!hide
\usepackage{fancyhdr}%!hide
\usepackage{extramarks}%!hide
\pagestyle{fancy}%!hide
\RenewDocumentCommand{\headrule}{}{
  \rule{0.05\headwidth}{0.2cm}%
  \rule[0.1cm]{0.9\headwidth}{\headrulewidth}%
  \rule{0.05\headwidth}{0.2cm}%
}

\RenewDocumentCommand{\headruleskip}{}{-0.2cm}

\begin{document}%!hide
\noindent %!hide
My age is the first number not
nameable in under twenty words.
\end{document} %!hide
\end{example}

While working on a longer document you may want to have several page styles for
different occasions. You may also notice that some commands (for example
\ci{chapter}, \ci{maketitle}) change the page style to \cargv{plain} whether
you want it or not. The solution to both of these problems is the
\begin{lscommand}
  \ci*{fancypagestyle}[name:m ! code:m]
\end{lscommand}
command. The \carg{name} is the name of page style to (re)define, while the
\carg{code} is the code to set up the page style. All page styles declared this
way use the \cargv{fancy} page style as its basis, so if empty \carg{code} is
given they will match \cargv{fancy} exactly.
 \begin{example}[standalone, paperheight=3cm]
\geometry{includehead, includefoot, headsep=.5em, footskip=1em} %!hide
\sloppy %!hide
\usepackage{fancyhdr}%!hide
\usepackage{extramarks}%!hide
\fancypagestyle{mine}{
  \fancyhead[L]{My style}
}
\pagestyle{mine}

\begin{document}%!hide
%!showbegin %!hide
Were you expecting a paradox here?
%!showend !hidebegin
\noindent
You may be disappointed.
\end{document}
\end{example}

\section{The Verbatim Package}

Earlier in this book, you got to know the \ei{verbatim}
\emph{environment}.  In this section, you are going to learn about the
\pai{verbatim} \emph{package}. The \pai{verbatim} package is basically
a re-implementation of the \ei{verbatim} environment that works around
some of the limitations of the original \ei{verbatim} environment.
This by itself is not spectacular, but the implementation of the
\pai{verbatim} package added new functionality, which is
why I am mentioning the package here. The \pai{verbatim}
package provides the

\begin{lscommand}
\ci{verbatiminput}\verb|{|\emph{filename}\verb|}|
\end{lscommand}

\noindent command, which allows you to include raw ASCII text into your
document as if it were inside a \ei{verbatim} environment.

As the \pai{verbatim} package is part of the `tools' bundle, you
should find it pre-installed on most systems. If you want to know more
about this package, make sure to read \cite{verbatim}.


\section{Installing Extra Packages}\label{sec:Packages}

Most \LaTeX{} installations come with a large set of pre-installed
style packages, but many more are available on the net. The main
place to look for style packages on the Internet is CTAN (\url{http://www.ctan.org/}).

Packages such as \pai{geometry}, \pai{hyphenat}, and many
others are typically made up of two files: a file with the extension
\texttt{.ins} and another with the extension \texttt{.dtx}. There
will often be a \texttt{readme.txt} with a brief description of the
package. You should of course read this file first.

In any event, once you have copied the package files onto your
machine, you still have to process them in a way that (a) tells your
\TeX\ distribution about the new style package and (b) gives you
the documentation.  Here's how you do the first part:

\begin{enumerate}
\item Run \LaTeX{} on the \texttt{.ins} file. This will
  extract a \eei{.sty} file.
\item Move the \eei{.sty} file to a place where your distribution
  can find it. Usually this is in your \texttt{\ldots/\emph{localtexmf}/tex/latex}
  subdirectory (Windows or OS/2 users should feel free to change the
  direction of the slashes).
\item Refresh your distribution's file-name database. The command
  depends on the \LaTeX distribution you use:
  \TeX{}live -- \texttt{texhash}; web2c -- \texttt{maktexlsr};
  MiK\TeX{} -- \texttt{initexmf -{}-update-fndb} or use the GUI.
\end{enumerate}

\noindent Now extract the documentation from the
\texttt{.dtx} file:

\begin{enumerate}
\item Run \hologo{XeLaTeX} on the \texttt{.dtx} file.  This will generate a
  \texttt{.pdf} file. Note that you may have to run \hologo{XeLaTeX}
  several times before it gets the cross-references right.
\item Check to see if \LaTeX\ has produced a \texttt{.idx} file
  among the various files you now have.
  If you do not see this file, then the documentation has no index. Continue
  with step~\ref{step:final}.
\item In order to generate the index, type the following:\\
        \fbox{\texttt{makeindex -s gind.ist \textit{name}}}\\
        (where \textit{name} stands for the main-file name without any
    extension).
 \item Run \LaTeX\ on the \texttt{.dtx} file once again. \label{step:next}

\item Last but not least, make a \texttt{.ps} or \texttt{.pdf}
  file to increase your reading pleasure.\label{step:final}

\end{enumerate}

Sometimes you will see that a \texttt{.glo}
(glossary) file has been produced. Run the following
command between
step~\ref{step:next} and~\ref{step:final}:

\noindent\texttt{makeindex -s gglo.ist -o \textit{name}.gls \textit{name}.glo}

\noindent Be sure to run \LaTeX\ on the \texttt{.dtx} one last
time before moving on to step~\ref{step:final}.


%%%%%%%%%%%%%%%%%%%%%%%%%%%%%%%%%%%%%%%%%%%%%%%%%%%%%%%%%%%%%%%%%
% Contents: Chapter on pdfLaTeX
% French original by Daniel Flipo 14/07/2004
%%%%%%%%%%%%%%%%%%%%%%%%%%%%%%%%%%%%%%%%%%%%%%%%%%%%%%%%%%%%%%%%%

\section{\LaTeX{} and PDF}\label{sec:pdftex}\index{PDF}
\secby{Daniel Flipo}{Daniel.Flipo@univ-lille1.fr}%
PDF is a portable \wi{hypertext} document format. Much as in a web page,
some words in the document are marked as hyperlinks. They link to other
places in the document or even to other documents. If you click
on such a hyperlink you get transported to the destination of the
link. In the context of \LaTeX{}, this means that all occurrences of
\ci{ref} and \ci{pageref} become hyperlinks. Additionally, the table
of contents, the index and all the other similar structures become
collections of hyperlinks.

Most web pages you find today are written in HTML \emph{(HyperText
  Markup Language)}. This format has two significant disadvantages
when writing scientific documents:
\begin{enumerate}
\item Including mathematical formulae into HTML documents is not
  generally supported. While there is a standard for it, most browsers
  used today do not support it, or lack the required fonts.
\item Printing HTML documents is possible, but the results vary widely
  between platforms and browsers. The results are miles removed from
  the quality we have come to expect in the \LaTeX{} world.
\end{enumerate}

There have been many attempts to create translators from \LaTeX{} to
HTML. Some were even quite successful in the sense that they are able
to produce legible web pages from a standard \LaTeX{} input file. But
all of them cut corners left and right to get the job done. As soon as
you start using more complex \LaTeX{} features and external packages
things tend to fall apart. Authors wishing to preserve the unique
typographic quality of their documents even when publishing on the web
turn to PDF \emph{(Portable Document Format)}, which preserves the
layout of the document and permits hypertext
navigation. Most modern browsers come with plugins that allow the
direct display of PDF documents.

All modern \TeX{} engines can generate PDF files out of the box. If you worked through this introduction until here you will already be familiar with the process.

\subsection{Hypertext Links}
\label{ssec:pdfhyperref}

The \pai{hyperref} adds two cool features to your \LaTeX{} PDF files:

\begin{enumerate}
\item The paper size is set according to your specification in the document class call.
\item All references in your document turn into hyperlinks.
\end{enumerate}

Just add \verb+\usepackage{hyperref}+ as the \emph{last} command into
the preamble of your document.

Many options are available to customize the behaviour of the
\pai{hyperref} package:
\begin{itemize}
\item either as a comma separated list after the pdftex option\\
  \verb+\usepackage{hyperref}+
\item or on individual lines with the command
  \verb+\hypersetup{+\emph{options}\verb+}+.
\end{itemize}

In the following
list the default values are written in an upright font.

\begin{flushleft}
\begin{description}
  \item [\texttt{bookmarks (=true,\textsl{false})}] show or hide the
    bookmarks bar when displaying the document
  \item [\texttt{unicode (=false,\textsl{true})}] allows the use of
    characters of non-Latin based languages in Acrobat's bookmarks
  \item [\texttt{pdftoolbar (=true,\textsl{false})}] show or hide
    Acrobat's toolbar
  \item [\texttt{pdfmenubar (=true,\textsl{false})}] show or hide
    Acrobat's menu
  \item [\texttt{pdffitwindow (=false,\textsl{true})}] adjust the
    initial magnification of the PDF when displayed
  \item [\texttt{pdftitle (=\{text\})}] define the title that gets
    displayed in the \texttt{Document Info} window of Acrobat
  \item [\texttt{pdfauthor (=\{text\})}] the name of the PDF's author
  \item [\texttt{pdfnewwindow (=false,\textsl{true})}] define whether a new
    window should be opened when a link leads out of the current
    document
  \item [\texttt{colorlinks (=false,\textsl{true})}] surround the
    links by colour frames (\texttt{false}) or colour the text of the links
    (\texttt{true}). The colour of these links can be configured
    using the following options (default colours are shown):
    \begin{description}
    \item [\texttt{linkcolor (=red)}] colour of internal
      links (sections, pages, etc.)
    \item [\texttt{citecolor (=green)}] colour of
      citation links (bibliography)
    \item [\texttt{filecolor (=magenta)}] colour of file
      links
    \item [\texttt{urlcolor (=cyan)}] colour of URL
      links (mail, web)
    \end{description}
\end{description}
\end{flushleft}

If you are happy with the defaults, use
\begin{code}
\begin{verbatim}
\usepackage{hyperref}
\end{verbatim}
\end{code}

To have the bookmark list open and links in colour
(the \texttt{=true} values are optional):
\begin{code}
\begin{verbatim}
\usepackage[bookmarks,colorlinks]{hyperref}
\end{verbatim}
\end{code}

When creating PDFs destined for printing, coloured links are not a
  good thing as they end up in gray in the final output, making it
  difficult to read. Use colour frames, which are not printed:
\begin{code}
\begin{verbatim}
\usepackage{hyperref}
\hypersetup{colorlinks=false}
\end{verbatim}
\end{code}
\noindent or make links black:
\begin{code}
\begin{verbatim}
\usepackage{hyperref}
\hypersetup{colorlinks,%
            citecolor=black,%
            filecolor=black,%
            linkcolor=black,%
            urlcolor=black,%
            pdftex}
\end{verbatim}
\end{code}

When you just want to provide information for the
  \texttt{Document Info} section of the PDF file:
\begin{code}
\begin{verbatim}
\usepackage[pdfauthor={Pierre Desproges},%
            pdftitle={Des femmes qui tombent},%
            pdftex]{hyperref}
\end{verbatim}
\end{code}

\vspace{\baselineskip}

In addition to the automatic hyperlinks for cross references, it is
possible to embed explicit links using
\begin{lscommand}
\ci{href}\verb|{|\emph{url}\verb|}{|\emph{text}\verb|}|
\end{lscommand}

The code
\begin{code}
\begin{verbatim}
The \href{http://www.ctan.org}{CTAN} website.
\end{verbatim}
\end{code}
produces the output ``\href{http://www.ctan.org}{CTAN}'';
a click on the word ``\textcolor{magenta}{CTAN}''
will take you to the CTAN website.

If the destination of the link is not a URL but a local file,
  use the \ci{href} command without the 'http://' bit:
\begin{verbatim}
  The complete document is \href{manual.pdf}{here}
\end{verbatim}
which produces the text ``The complete document is \textcolor{cyan}{here}''.
A click on the word
``\textcolor{cyan}{here}''
will open the file \texttt{manual.pdf}. (The filename is relative to
the location of the current document).

The author of an article might want her readers to easily send
  email messages by using the \ci{href} command inside the \ci{author}
  command on the title page of the document:
\begin{code}
\begin{verbatim}
\author{Mary Oetiker $<$\href{mailto:mary@oetiker.ch}%
       {mary@oetiker.ch}$>$
\end{verbatim}
\end{code}
Note that I have put the link so that my email address appears not only
in the link but also on the page itself. I did this because the
link\\
\verb+\href{mailto:mary@oetiker.ch}{Mary Oetiker}+\\
would
work well within Acrobat, but once the page is printed the email address
would not be visible anymore.


\subsection{Problems with Links}

Messages like the following:
\begin{verbatim}
! pdfTeX warning (ext4): destination with the same
  identifier (name{page.1}) has been already used,
  duplicate ignored
\end{verbatim}
appear when a counter gets reinitialized, for example by using
the command \ci{mainmatter} provided by the \texttt{book} document class. It
resets the page number counter to~1 prior to the first chapter of the
book. But as the preface of the book also has a page number~1 all
links to ``page 1'' would not be unique anymore, hence the notice
that ``\verb+duplicate+ has been \verb+ignored+.''

The counter measure consists of putting \texttt{plainpages=false} into
the hyperref options. This unfortunately only helps with the page
counter.
An even more radical solution is to use the option\\
\texttt{hypertexnames=false}, but this will cause the page links in
the index to stop working.

\subsection{Problems with Bookmarks}

The text displayed by bookmarks does not always look like you expect
it to look. Because bookmarks are ``just text,'' fewer
characters are available for bookmarks than for normal \LaTeX{} text.
Hyperref will normally notice such problems and put up a warning:
\begin{code}
\begin{verbatim}
Package hyperref Warning:
Token not allowed in a PDFDocEncoded string:
\end{verbatim}
\end{code}
Work around this problem by providing a text string for
the bookmarks, which replaces the offending text:
\begin{lscommand}
\ci{texorpdfstring}\verb|{|\emph{\TeX{} text}\verb|}{|\emph{Bookmark Text}\verb|}|
\end{lscommand}


Math expressions are a prime candidate for this kind of problem:
\begin{code}
\begin{verbatim}
\section{\texorpdfstring{$E=mc^2$}%
        {E = mc ** 2}}
\end{verbatim}
\end{code}
which turns \verb+\section{$E=mc^2$}+ to ``E = mc ** 2'' in the bookmark area.

If you write your document in Unicode and use the \verb+unicode+ option for
the \pai{hyperref} package to use Unicode characters in bookmarks, this
will give you a much larger selection of characters to pick from when
when using \ci{texorpdfstring}.

\section{Working with \hologo{XeLaTeX} and PDF}
\label{sec:xetex}\index{PDF}\index{XeTeX@\hologo{XeTeX}}\index{XeLaTeX@\hologo{XeLaTeX}}

\secby{Axel Kielhorn}{A.Kielhorn@web.de}%
Most of the things said in the previous section are valid for \hologo{XeLaTeX} as well.

There is a Wiki at \url{http://wiki.xelatex.org/doku.php} that collects
information relevant to \hologo{XeTeX} and \hologo{XeLaTeX}.

\subsection{The Fonts} \label{sec:fontspec}

In addition to the normal \texttt{tfm} based fonts, \hologo{XeLaTeX} is able
to use any font known to the operating system. If you have the \texttt{Linux
Libertine} fonts installed, you can simply say

\begin{code}
\begin{verbatim}
\usepackage{fontspec}
\setmainfont[Ligatures=TeX]{Linux Libertine}
\end{verbatim}
\end{code}
%
in the preamble. This will normally detect the italic and bold versions as
well, so \verb|\textit| and \verb|\textbf| will work as usual. When the
font is using OpenType technology you have access to many features which
required switching to a separate font or using virtual fonts in the past.
The main feature is the extended character set; a font may contain Latin,
Greek and Cyrillic characters and the corresponding ligatures.

Many fonts contain at least two kinds of numerals, the normal lining
numerals and so called old style (or lower case) numerals, which partly
extend below the baseline. They may contain proportional numerals (the ``1''
takes less space than the ``0'') or monospaced numerals which are suitable
for tables.

\begin{code}
\begin{verbatim}
\newfontfamily\LLln[Numbers=Lining]{(font)}
\newfontfamily\LLos[Numbers=OldStyle]{(font)}
\newfontfamily\LLlnm[
  Numbers=Lining,Numbers=Monospaced
]{(font)}
\newfontfamily\LLosm[
  Numbers=OldStyle,Numbers=Monospaced
]{(font)}
\end{verbatim}
\end{code}

Almost all OpenType fonts contain the standard ligatures (fl fi ffi) but
there are also some rare or historical ligatures like st, ct and tz. You may
not want to use them in a technical report but they are fine for a novel. To
enable these ligatures use either of the following lines:

\begin{code}
\begin{verbatim}
\setmainfont[Ligatures=Rare]{(font)}
\setmainfont[Ligatures=Historic]{(font)}
\setmainfont[Ligatures=Historic,Ligatures=Rare]{(font)}
\end{verbatim}
\end{code}

Not every font contains both sets of ligature, consult the font
documentation or just try it out. Sometimes these ligatures are language
dependent; for example a ligature used in Polish (fk) is not used in English. You have
to add
\begin{code}
\begin{verbatim}
\setmainfont[Language=Polish]{(font)}
\end{verbatim}
\end{code}
to enable the Polish ligatures.

Some fonts (like the commercial Adobe Garamond Premier Pro) contain
alternative glyphs that are activated by default in \hologo{XeLaTeX}
distributed with \TeX Live~2010\footnote{The behavior has changed with this
version, it was off by default in earlier releases.}. The result is a stylish
``Q'' with a descender reaching below the following ``u''. To disable this
feature you have to define the font with disabled contextuals:

\begin{code}
\begin{verbatim}
\setmainfont[Contextuals=NoAlternate]{(font)}
\end{verbatim}
\end{code}

To learn about fonts in \hologo{XeLaTeX} read the \pai{fontspec} manual.

\subsubsection{Where do I get OpenType fonts?}

If you have \texttt{TeXLive} installed, you already have some at
\url{.../texmf-dist/fonts/opentype}, just install them in your operating
system. This collection does not include \texttt{DejaVu}, which is available
at \url{http://dejavu-fonts.org/}.

Make sure that each font is only installed \emph{once}, otherwise
interesting results may happen.

You can use every font installed on your computer, but remember that other
users may not have these fonts. The Zapfino font used in the \pai{fontspec}
manual is included in Mac OSX, but is not available on Windows
computers.\footnote{A commercial version of the font called Zapfino Extra is
available.}

\subsubsection{Entering Unicode Characters}

The number of characters in a font has grown but the number of keys on a
regular keyboard has not. So, how do I enter non-ASCII characters?

If you write a large amount of text in a foreign language, you can install a
keyboard for that language and print out the character positions. (Most
operatings system have some sort of virtual keyboard, just make a
screenshot.)

If you rarely need an exotic character, you can simply pick it in the
character palette.

Some environments (e.\,g. the X Window System) offer many methods to enter
non-ASCII characters. Some editors (e.\,g. Vim and Emacs) offer ways to
enter these characters. Read the manual for the tools you are using.

\subsection{Compatibility Between \hologo{XeLaTeX} and \hologo{pdfLaTeX}}

There are a few things that are different between \hologo{XeLaTeX} and \hologo{pdfLaTeX}.

\begin{itemize}
	\item A \hologo{XeLaTeX} document has to be written in
	Unicode (UTF-8) while \hologo{pdfLaTeX} may use different input encodings.
\item The \pai{microtype} packages does not work with \hologo{XeLaTeX} yet,
	support for character protrusion is already under development.
\item Everything font related has to be reviewed. (Unless you want to stick
	to Latin Modern.)
\end{itemize}

\section{Creating Presentations}
\label{sec:beamer}
\secby{Daniel Flipo}{Daniel.Flipo@univ-lille1.fr}
You can present the results of your scientific work on a blackboard,
with transparencies, or directly from your laptop using some
presentation software.

\wi{pdf\LaTeX} combined with the \pai{beamer} class allows you to
create presentations in PDF, looking much like something you might be
able to generate with LibreOffice or PowerPoint if you had a very good day, but much
more portable because PDF readers are available on many more
systems.

The \pai{beamer} class uses \pai{graphicx}, \pai{color} and
\pai{hyperref} with options adapted to screen presentations.
%La figure~\ref{fig:pdfscr} contient un exemple de fichier minimal
%compiler avec \wi{pdf\LaTeX} et le
%rsultat produit.

% cran captur par ImageMagick (man ImageMagick) fonction  import
% et convertie en jpg toujours par ImageMagick.

\begin{figure}[htbp]
\begin{verbatim}
\documentclass[10pt]{beamer}
\mode<beamer>{%
  \usetheme[hideothersubsections,
            right,width=22mm]{Goettingen}
}

\title{Simple Presentation}
\author[D. Flipo]{Daniel Flipo}
\institute{U.S.T.L. \& GUTenberg}
\titlegraphic{\includegraphics[width=20mm]{USTL}}
\date{2005}

\begin{document}

\begin{frame}<handout:0>
  \titlepage
\end{frame}

\section{An Example}

\begin{frame}
  \frametitle{Things to do on a Sunday Afternoon}
  \begin{block}{One could \ldots}
    \begin{itemize}
      \item walk the dog\dots \pause
      \item read a book\pause
      \item confuse a cat\pause
    \end{itemize}
  \end{block}
  and many other things
\end{frame}
\end{document}
\end{verbatim}
  \caption{Sample code for the \pai{beamer} class}
  \label{fig:code-beamer}
\end{figure}

When you compile the code presented in figure~\ref{fig:code-beamer}
with \wi{pdf\LaTeX} you get a PDF file with a title page and a second page
showing several items that will be revealed one at a time as you step
though your presentation.

One of the advantages of the beamer class is that it produces a PDF
file that is directly usable without first going through a \PSi{}
stage like \pai{prosper} or requiring additional post processing like
presentations created with the \pai{ppower4} package.

With the \pai{beamer} class you can produce several versions (modes) of your
document from the same input file. The input file may contain special
instructions for the different modes in angular brackets. The
following modes are available:
\begin{description}
\item[beamer] for the presentation PDF
  discussed above.
\item[trans] for transparencies.
\item[handout] for the printed version.
\end{description}
The default mode is \texttt{beamer}, change it by setting a
different mode as a global option, like
\verb|\documentclass[10pt,handout]{beamer}| to print the handouts for
example.

The look of the screen presentation depends on the theme you choose. Pick one of the themes shipped with the beamer class or
create your own. See the beamer class documentation in
\texttt{beameruserguide.pdf} for more information on this.

Let's have a closer look at the code in figure~\ref{fig:code-beamer}.

For the screen version of the presentation \verb|\mode<beamer>| we
have chosen the \emph{Goettingen} theme to show a navigation panel
integrated into the table of contents. The options allow us to choose the
size of the panel (22~mm in this case) and its position (on the right
side of the body text). The option \emph{hideothersubsections}, shows
the chapter titles, but only the subsections of the present
chapter. There are no special settings for \verb|\mode<trans>| and
\verb|\mode<handout>|. They appear in their standard layout.

The commands \verb|\title{}|, \verb|\author{}|, \verb|\institute{}|,
and\\ \verb|\titlegraphic{}| set the content of the title page. The
optional arguments of \verb|\title[]{}| and \verb|\author[]{}|
let you specify a special version of the title and the author
name to be displayed on the panel of the \emph{Goettingen} theme.

The titles and subtitles in the panel are created with normal
\verb|\section{}| and \verb|\subsection{}| commands that you place
\emph{outside} the \ei{frame} environment.

The tiny navigation icons at the bottom of the screen also allow to
navigate the document. Their presence is not dependent on the theme
you choose.

The contents of each slide or screen has to be placed inside a
\ei{frame} environment. There is an optional argument in angular
brackets (\verb|<| and \verb|>|), it allows us to suppress a particular
frame in one of the versions of the presentation. In the example the
first page would not be shown in the handout version due to the
\verb|<handout:0>| argument.

It is highly recommended to set a title for each slide apart from the
title slide. This is done with the command \verb|\frametitle{}|. If a
subtitle is necessary use the \ei{block} environment as shown
in the example. Note that the sectioning commands \verb|\section{}|
and \verb|\subsection{}| do not produce output on the slide proper.

The command \verb|\pause| in the itemize environment lets you reveal
the items one by one. For other presentation effects check out the
commands \verb|\only|, \verb|\uncover|, \verb|\alt| and
\verb|\temporal|. In many place it is possible to use angular brackets to
further customize the presentation.

In any case make sure to read through the beamer class documentation
\texttt{beameruserguide.pdf} to get a complete picture of what is in
store for you. This package is being actively developed, check out their website
to get the latest information. (\href{http://latex-beamer.sourceforge.net/}{http://latex-beamer.sourceforge.net/})



% Local Variables:
% TeX-master: "lshort2e"
% mode: latex
% mode: flyspell
% End:
