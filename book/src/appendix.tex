% The Not So Short Introduction to LaTeX
%
% Copyright (C) 1995--2022 Tobias Oetiker, Marcin Serwin, Hubert Partl,
% Irene Hyna, Elisabeth Schlegl and Contributors.
%
% This document is free software: you can redistribute it and/or modify it
% under the terms of the GNU General Public License as published by the Free
% Software Foundation, either version 3 of the License, or (at your option) any
% later version.
%
% This document is distributed in the hope that it will be useful, but WITHOUT
% ANY WARRANTY; without even the implied warranty of MERCHANTABILITY or FITNESS
% FOR A PARTICULAR PURPOSE.  See the GNU General Public License for more
% details.
%
% You should have received a copy of the GNU General Public License along with
% this document.  If not, see <https://www.gnu.org/licenses/>.

% !TEX root = ./lshort.tex
\chapter{Installing \LaTeX}\label{installinglatex}
\begin{intro}
  Knuth published the source to \TeX{} back in a time when nobody knew
  about Open Source and/or Free Software. The License that comes with \TeX{}
  lets you do whatever you want with the source, but you can only call the
  result of your work \TeX{} if the program passes a set of tests Knuth has
  also provided. This has lead to a situation where we have free \TeX{}
  implementations for almost every Operating System under the sun. This chapter
  will give some hints on what to install on Linux, macOS and Windows, to
  get a working \TeX{} setup.
\end{intro}

\section{What to Install}

To use \LaTeX{} on any computer system, you need several programs.

\begin{enumerate}

  \item The \TeX{}/\LaTeX{} program for processing your \LaTeX{} source files
        into typeset PDF documents.

  \item A text editor for editing your \LaTeX{} source files. Some products even let
        you start the \LaTeX{} program from within the editor.

  \item A PDF viewer program for previewing and printing your
        documents.

  \item A program to handle \PSi{} files and images for inclusion into
        your documents.

\end{enumerate}

For every platforms there are several programs that fit the requirements above.
Here we just tell about the ones we know, like and have some experience
with.

\section{Cross Platform Editor}\label{sec:texmaker}

While \TeX{} is available on many different computing platforms, \LaTeX{}
editors have long been highly platform specific.

Over the past few years I have come to like Texmaker quite a lot.
Apart from being very a useful editor with integrated pdf-preview and syntax
high-lighting, it has the advantage of running on Windows, Mac and
Unix/Linux equally well.  See~\cite{texmaker} for
further information.  There is also a forked version of Texmaker called
TeXstudio~\cite{texstudio}.  It also seems well
maintained and is also available for all three major platforms.

You will find some platform specific editor suggestions in the OS sections
below.

\section{\TeX{} on macOS}

\subsection{\TeX{} Distribution}

Just download \wi{MacTeX}~\cite{mactex}. It is a
pre-compiled \LaTeX{} distribution for macOS\@. \wi{MacTeX} provides a full \LaTeX{}
installation plus a number of additional tools.

\subsection{macOS \TeX{} Editor}

If you are not happy with our cross-platform suggestion Texmaker (\autoref{sec:texmaker}).

The most popular open source editor for \LaTeX{} on the mac seems to be
\TeX{}shop~\cite{texshop}. It
is also contained in the \wi{MacTeX} distribution.

Recent \TeX{}Live distributions contain the \TeX{}works editor~\cite{texworks}
which is a multi-platform editor based on the \TeX{}Shop
design. Since \TeX{}works uses the Qt toolkit, it is available on any platform
supported by this toolkit (macOS, Windows, Linux).

\subsection{Treat yourself to \wi{PDFView}}

Use PDFView~\cite{pdfview} for viewing PDF files generated by \LaTeX{}, it integrates tightly
with your \LaTeX{} text editor. After installing, open
PDFViews preferences dialog and make sure that the \emph{automatically reload
  documents} option is enabled and that PDFSync support is set appropriately.

\section{\TeX{} on Windows}

\subsection{Getting \TeX{}}

First, get a copy of the excellent MiK\TeX\index{MiKTeX@MiK\TeX}~\cite{miktex} distribution.
It contains all the basic programs and files
required to compile \LaTeX{} documents.  The coolest feature in my eyes, is
that MiK\TeX{} will download missing \LaTeX{} packages on the fly and install them
magically while compiling a document. Alternatively you can also use
the TeXlive~\cite{texlive} distribution which exists for Windows, Unix and Mac OS to
get your base setup going.

\subsection{A \LaTeX{} editor}

If you are not happy with our cross-platform suggestion Texmaker
(\autoref{sec:texmaker}).

\wi{TeXnicCenter}~\cite{texniccenter} uses many concepts from the programming-world to provide a nice and
efficient \LaTeX{} writing environment in Windows. TeXnicCenter integrates nicely with
MiKTeX.

Recent \TeX{}Live distributions contain the \TeX{}works Editor~\cite{texworks}.
It supports Unicode and requires at least Windows XP\@.

\subsection{Document Preview}

You will most likely be using Yap for DVI preview as it gets installed with
MikTeX. For PDF you may want to look at Sumatra
PDF~\cite{sumatrapdf}. I mention Sumatra PDF
because it lets you jump from any position in the pdf document back into
corresponding position in your source document.

\subsection{Working with graphics}

Working with high quality graphics in \LaTeX{} means that you have to use
\EPSi{} (eps) or PDF as your picture format. The program that helps you
deal with this is called \wi{GhostScript}~\cite{ghostscript}. It comes with its
own front-end \wi{GhostView}.

If you deal with bitmap graphics (photos and scanned material), you may want
to have a look at the open source Photoshop alternative \wi{Gimp}~\cite{gimp}.

\section{\TeX{} on Linux}

If you work with Linux, chances are high that \LaTeX{} is already installed
on your system, or at least available on the installation source you used to
setup. Use your package manager to install the following packages:

\begin{itemize}
  \item texlive --- the base \TeX{}/\LaTeX{} setup.
  \item emacs (with AUCTeX) --- an editor that integrates tightly with \LaTeX{} through the add-on AUCTeX package.
  \item ghostscript --- a \PSi{} preview program.
  \item xpdf and acrobat --- a PDF preview program.
  \item imagemagick --- a free program for converting bitmap images.
  \item gimp --- a free Photoshop look-a-like.
  \item inkscape --- a free illustrator/corel draw look-a-like.
\end{itemize}

If you are looking for a more windows like graphical editing environment,
check out Texmaker. See \autoref{sec:texmaker}.

Most Linux distros insist on splitting up their \TeX{} environments into a
large number of optional packages, so if something is missing after your
first install, go check again.
