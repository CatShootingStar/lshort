%%%%%%%%%%%%%%%%%%%%%%%%%%%%%%%%%%%%%%%%%%%%%%%%%%%%%%%%%%%%%%%%%
% Contents: Things you need to know
% $Id$
%%%%%%%%%%%%%%%%%%%%%%%%%%%%%%%%%%%%%%%%%%%%%%%%%%%%%%%%%%%%%%%%%

\chapter{Things You Need to Know}
\begin{intro}
The first part of this chapter presents a short
overview of the philosophy and history of \LaTeXe. The second part
focuses on the basic structures of a \LaTeX{} document.
After reading this chapter, you should have a rough knowledge
of how \LaTeX{} works, which you will need to understand the rest
of this book.
\end{intro}

\section{A Bit of History}
\subsection{\TeX}

\TeX{} is a computer program created by \index{Knuth, Donald E.}Donald
E. Knuth \cite{texbook}. It is aimed at typesetting text and
mathematical formulae. Knuth started writing the \TeX{} typesetting
engine in 1977 to explore the potential of the digital printing
equipment that was beginning to infiltrate the publishing industry at
that time, especially in the hope that he could reverse the trend of
deteriorating typographical quality that he saw affecting his own
books and articles. \TeX{} as we use it today was released in 1982,
with some slight enhancements added in 1989 to better support 8-bit
characters and multiple languages. \TeX{} is renowned for being
extremely stable, for running on many different kinds of computers,
and for being virtually bug free. The version number of \TeX{} is
converging to $\pi$ and is now at $3.14159265$.


\TeX{} is pronounced ``Tech,'' with a ``ch'' as in the German word
``Ach''\footnote{In German there are actually two pronunciations for ``ch''
and one might assume that the soft ``ch'' sound from ``Pech'' would be a
more appropriate. Asked about this, Knuth wrote in the German Wikipedia:
\emph{I do not get angry when people pronounce \TeX{} in their favorite way
\ldots{} and in Germany many use a soft ch because the X follows the vowel
e, not the harder ch that follows the vowel a. In Russia, `tex' is a very
common word, pronounced `tyekh'. But I believe the most proper pronunciation
is heard in Greece, where you have the harsher ch of ach and Loch.}}
or in the Scottish ``Loch.'' The ``ch'' originates from the Greek
alphabet where X is the letter ``ch'' or ``chi''. \TeX{} is also the first syllable
of the Greek word technique. In an ASCII environment, \TeX{}
becomes \texttt{TeX}.

\subsection{\LaTeX}

\LaTeX{} enables authors to typeset and print their work at the highest
typographical quality, using a predefined, professional layout. \LaTeX{} was
originally written by \index{Lamport, Leslie}Leslie Lamport~\cite{manual}.
It uses the \TeX{} formatter as its typesetting engine. These days \LaTeX{}
is maintained by \index{The \LaTeX{} Project}the \LaTeX{} Project.

%In 1994 the \LaTeX{} package was updated by the \index{LaTeX3@\LaTeX
%  3}\LaTeX 3 team, led by \index{Mittelbach, Frank}Frank Mittelbach,
%to include some long-requested improvements, and to re\-unify all the
%patched versions which had cropped up since the release of
%\index{LaTeX 2.09@\LaTeX{} 2.09}\LaTeX{} 2.09 some years earlier. To
%distinguish the new version from the old, it is called \index{LaTeX
%2e@\LaTeXe}\LaTeXe. This documentation deals with \LaTeXe. These days you
%might be hard pressed to find the venerable \LaTeX{} 2.09 installed
%anywhere.

\LaTeX{} is pronounced ``Lay-tech'' or ``Lah-tech.'' If you refer to
\LaTeX{} in an ASCII environment, you type \texttt{LaTeX}.
\LaTeXe{} is pronounced ``Lay-tech two e'' and typed \texttt{LaTeX2e}.

%Figure~\ref{components} above % on page \pageref{components}
%shows how \TeX{} and \LaTeXe{} work together. This figure is taken from
%\texttt{wots.tex} by Kees van der Laan.

%\begin{figure}[btp]
%\begin{lined}{0.8\textwidth}
%\begin{center}
%\input{kees.fig}
%\end{center}
%\end{lined}
%\caption{Components of a \TeX{} System.} \label{components}
%\end{figure}

\section{Basics}

\subsection{Author, Book Designer, and Typesetter}

To publish something, authors give their typed manuscript to a
publishing company. One of their book designers then
decides the layout of the document (column width, fonts, space before
and after headings,~\ldots). The book designer writes his instructions
into the manuscript and then gives it to a typesetter, who typesets the
book according to these instructions.

A human book designer tries to find out what the author had in mind
while writing the manuscript. He decides on chapter headings,
citations, examples, formulae, etc.\ based on his professional
knowledge and from the contents of the manuscript.

In a \LaTeX{} environment, \LaTeX{} takes the role of the book
designer and uses \TeX{} as its typesetter. But \LaTeX{} is ``only'' a
program and therefore needs more guidance. The author has to provide
additional information to describe the logical structure of his
work. This information is written into the text as ``\LaTeX{}
commands.''

This is quite different from the \wi{WYSIWYG}\footnote{What you see is
  what you get.} approach that most modern word processors, such as
\emph{MS Word} or \emph{LibreOffice}, take. With these
applications, authors specify the document layout interactively while
typing text into the computer. They can see on the
screen how the final work will look when it is printed.

When using \LaTeX{} it is not normally possible to see the final output
while typing the text, but the final output can be previewed on the
screen after processing the file with \LaTeX. Then corrections can be
made before actually sending the document to the printer.

\subsection{Layout Design}

Typographical design is a craft. Unskilled authors often commit
serious formatting errors by assuming that book design is mostly a
question of aesthetics---``If a document looks good artistically,
it is well designed.'' But as a document has to be read and not hung
up in a picture gallery, the readability and understandability is
much more important than the beautiful look of it.
Examples:
\begin{itemize}
\item The font size and the numbering of headings have to be chosen to make
  the structure of chapters and sections clear to the reader.
\item The line length has to be short enough not to strain
  the eyes of the reader, while long enough to fill the page
  beautifully.
\end{itemize}

With \wi{WYSIWYG} systems, authors often generate aesthetically
pleasing documents with very little or inconsistent structure.
\LaTeX{} prevents such formatting errors by forcing the author to
declare the \emph{logical} structure of his document. \LaTeX{} then
chooses the most suitable layout.

\subsection{Advantages and Disadvantages}

When people from the \wi{WYSIWYG} world meet people who use \LaTeX{},
they often discuss ``the \wi{advantages of \LaTeX{}} over a normal
word processor'' or the opposite.  The best thing to do when such
a discussion starts is to keep a low profile, since such discussions
often get out of hand. But sometimes there is no escaping \ldots

\medskip\noindent So here is some ammunition. The main advantages
of \LaTeX{} over normal word processors are the following:

\begin{itemize}

\item Professionally crafted layouts are available, which make a
  document really look as if ``printed.''
\item The typesetting of mathematical formulae is supported in a
  convenient way.
\item Users only need to learn a few easy-to-understand commands
  that specify the logical structure of a document. They almost never
  need to tinker with the actual layout of the document.
\item Even complex structures such as footnotes, references, table of
  contents, and bibliographies can be generated easily.
\item Free add-on packages exist for many typographical tasks not directly supported by basic
  \LaTeX. For example, packages are
  available to include \PSi{} graphics or to typeset
  bibliographies conforming to exact standards. Many of these add-on
  packages are described in \companion.
\item \LaTeX{} encourages authors to write well-structured texts,
  because this is how \LaTeX{} works---by specifying structure.
\item \TeX, the formatting engine of \LaTeXe, is highly portable and free.
  Therefore the system runs on almost any hardware platform
  available.

%
% Add examples ...
%
\end{itemize}

\medskip

\noindent\LaTeX{} also has some disadvantages, and I guess it's a bit
difficult for me to find any sensible ones, though I am sure other people
can tell you hundreds \texttt{;-)}

\begin{itemize}
\item \LaTeX{} does not work well for people who have sold their
  souls \ldots
\item Although some parameters can be adjusted within a predefined
  document layout, the design of a whole new layout is difficult and
  takes a lot of time.\footnote{Rumour says that this is one of the
    key elements that will be addressed in the upcoming \LaTeX 3
    system.}\index{LaTeX3@\LaTeX 3}
\item It is very hard to write unstructured and disorganized documents.
\item Your hamster might, despite some encouraging first steps, never be
able to fully grasp the concept of Logical Markup.
\end{itemize}

\section{\LaTeX{} Input Files}

The input for \LaTeX{} is a plain text file. On Unix/Linux text files are
pretty common. On windows, one would use Notepad to create a text file. It
contains the text of the document, as well as the commands that tell
\LaTeX{} how to typeset the text. If you are working with a \LaTeX{} IDE, it will contain a program for creating
\LaTeX{} input files in text format.

\subsection{Spaces}

``Whitespace'' characters, such as blank or tab, are
treated uniformly as ``\wi{space}'' by \LaTeX{}. \emph{Several
  consecutive} \wi{whitespace} characters are treated as \emph{one}
``space''.  Whitespace at the start of a line is generally ignored, and
a single line break is treated as ``whitespace''.
\index{whitespace!at the start of a line}

An empty line between two lines of text defines the end of a
paragraph. \emph{Several} empty lines are treated the same as
\emph{one} empty line. The text below is an example. On the left hand
side is the text from the input file, and on the right hand side is the
formatted output.

\begin{example}
It does not matter whether you
enter one or several     spaces
after a word.

An empty line starts a new
paragraph.
\end{example}

\subsection{Special Characters}

The following symbols are \wi{reserved characters} that either have a
special meaning under \LaTeX{} or are not available in all the fonts.
If you enter them directly in your text, they will normally not print,
but rather coerce \LaTeX{} to do things you did not intend.
\begin{code}
\verb.#  $  %  ^  &  _  {  }  ~  \ . %$
\end{code}

As you will see, these characters can be used in your documents all
the same by using a prefix backslash:

\begin{example}
\# \$ \% \^{} \& \_ \{ \} \~{}
\textbackslash
\end{example}

The other symbols and many more can be printed with special commands
in mathematical formulae or as accents. The backslash character
\textbackslash{} can \emph{not} be entered by adding another backslash
in front of it (\verb|\\|); this sequence is used for
line breaking. Use the \ci{textbackslash} command instead.

\subsection{\LaTeX{} Commands}

\LaTeX{} \wi{commands} are case sensitive, and take one of the following
two formats:

\begin{itemize}
\item They start with a \wi{backslash} \verb|\| and then have a name
 consisting of letters only. Command names are terminated by a
 space, a number or any other `non-letter.'
\item They consist of a backslash and exactly one non-letter.
\item Many commands exist in a `starred variant' where a star is appended
to the command name.
\end{itemize}

%
% \\* doesn't comply !
%

%
% Can \3 be a valid command ? (jacoboni)
%
\label{whitespace}
\hyphenation{white-spaces white-space}
\LaTeX{} ignores whitespace after commands. If you want to get a
\index{whitespace!after commands}space after a command, you have to
put either an empty parameter \verb|{}| and a blank or a special spacing command after the
command name. The empty parameter \verb|{}| stops \LaTeX{} from eating up all the white space after
the command name.

\begin{example}
New \TeX users may miss whitespaces
after a command. % renders wrong
Experienced \TeX{} users are
\TeX perts, and know how to use
whitespaces. % renders correct
\end{example}

Some commands require a \wi{parameter}, which has to be given between
\wi{curly braces} \verb|{ }| after the command name. Some commands take
\wi{optional parameters}, which are inserted after the command name in
\wi{square brackets}~\verb|[ ]|.
\begin{code}
\verb|\|\textit{command}\verb|[|\textit{optional parameter}\verb|]{|\textit{parameter}\verb|}|
\end{code}
The next examples use some \LaTeX{}
commands. Don't worry about them; they will be explained later.

\begin{example}
You can \textsl{lean} on me!
\end{example}
\begin{example}
Please, start a new line
right here!\newline
Thank you!
\end{example}

\subsection{Comments}
\index{comments}

When \LaTeX{} encounters a \verb|%| character while processing an input file,
it ignores the rest of the present line, the line break, and all
whitespace at the beginning of the next line.

This can be used to write notes into the input file, which will not show up
in the printed version.

\begin{example}
This is an % stupid
% Better: instructive <----
example: Supercal%
              ifragilist%
    icexpialidocious
\end{example}

The \texttt{\%} character can also be used to split long input lines where no
whitespace or line breaks are allowed.

For longer comments you could use the \ei{comment} environment
provided by the \pai{verbatim} package. Add the
line \verb|\usepackage{verbatim}| to the preamble of your document as
explained below to use this command.

\begin{example}
This is another
\begin{comment}
rather stupid,
but helpful
\end{comment}
example for embedding
comments in your document.
\end{example}

Note that this won't work inside complex environments, like math for example.

\section{Input File Structure}
\label{sec:structure}
When \LaTeXe{} processes an input file, it expects it to follow a
certain \wi{structure}. Thus every input file must start with the
command
\begin{code}
\verb|\documentclass{...}|
\end{code}
This specifies what sort of document you intend to write. After that,
add commands to influence the style of the whole
document, or load \wi{package}s that add new
features to the \LaTeX{} system. To load such a package you use the
command
\begin{code}
\verb|\usepackage{...}|
\end{code}

When all the setup work is done,\footnote{The area between \texttt{\bs
    documentclass} and \texttt{\bs
    begin$\mathtt{\{}$document$\mathtt{\}}$} is called the
  \emph{\wi{preamble}}.} you start the body of the text with the
command

\begin{code}
\verb|\begin{document}|
\end{code}

Now you enter the text mixed with some useful \LaTeX{} commands.  At
the end of the document you add the
\begin{code}
\verb|\end{document}|
\end{code}
command, which tells \LaTeX{} to call it a day. Anything that
follows this command will be ignored by \LaTeX.

Figure~\ref{mini} shows the contents of a minimal \LaTeXe{} file. A
slightly more complicated \wi{input file} is given in
Figure~\ref{document}.

\begin{figure}[!bp]
\begin{lined}{6cm}
\begin{verbatim}
\documentclass{article}
\begin{document}
Small is beautiful.
\end{document}
\end{verbatim}
\end{lined}
\caption{A Minimal \LaTeX{} File.} \label{mini}
\end{figure}

\begin{figure}[!bp]
\begin{lined}{10cm}
\begin{verbatim}
\documentclass[a4paper,11pt]{article}
% define the title
\author{H.~Partl}
\title{Minimalism}
\begin{document}
% generates the title
\maketitle
% insert the table of contents
\tableofcontents
\section{Some Interesting Words}
Well, and here begins my lovely article.
\section{Good Bye World}
\ldots{} and here it ends.
\end{document}
\end{verbatim}
\end{lined}
\caption[Example of a Realistic Journal Article.]{Example of a Realistic
Journal Article. Note that all the commands you see in this example will be
explained later in the introduction.} \label{document}

\end{figure}

\section{A Typical Command Line Session}

I bet you must be dying to try out the neat small \LaTeX{} input file
shown on page \pageref{mini}. Here is some help:
\LaTeX{} itself comes without a GUI or
fancy buttons to press. It is just a program that crunches away at your
input file. Some \LaTeX{} installations feature a graphical front-end where
there is a \LaTeX{} button to start compiling your input file. On other systems
there might be some typing involved, so here is how to coax \LaTeX{} into
compiling your input file on a text based system. Please note: this
description assumes that a working \LaTeX{} installation already sits on
your computer.\footnote{This is the case with most well groomed Unix
Systems, and \ldots{} Real Men use Unix, so \ldots{} \texttt{;-)}}

\begin{enumerate}
\item

  Edit/Create your \LaTeX{} input file. This file must be plain ASCII
  text.  On Unix all the editors will create just that. On Windows you
  might want to make sure that you save the file in ASCII or
  \emph{Plain Text} format.  When picking a name for your file, make
  sure it bears the extension \eei{.tex}.

\item

Open a shell or cmd window, \texttt{cd} to the directory where your input file is located and run \LaTeX{} on your input file. If successful you will end up with a
\texttt{.pdf} file. It may be necessary to run \LaTeX{} several times to get
the table of contents and all internal references right. When your input
file has a bug \LaTeX{} will tell you about it and stop processing your
input file. Type \texttt{ctrl-D} to get back to the command line.
\begin{lscommand}
\verb+xelatex foo.tex+
\end{lscommand}

\end{enumerate}


\section{The Layout of the Document}

\subsection {Document Classes}\label{sec:documentclass}

The first information \LaTeX{} needs to know when processing an
input file is the type of document the author wants to create. This
is specified with the \ci{documentclass} command.
\begin{lscommand}
\ci{documentclass}\verb|[|\emph{options}\verb|]{|\emph{class}\verb|}|
\end{lscommand}
\noindent Here \emph{class} specifies the type of document to be created.
Table~\ref{documentclasses} lists the document classes explained in
this introduction. The \LaTeXe{} distribution provides additional
classes for other documents, including letters and slides.  The
\emph{\wi{option}s} parameter customizes the behavior of the document
class. The options have to be separated by commas. The most common options for the standard document
classes are listed in
Table~\ref{options}.


\begin{table}[!bp]
\caption{Document Classes.} \label{documentclasses}
\begin{lined}{\textwidth}
\begin{description}

\item [\normalfont\texttt{article}] for articles in scientific journals, presentations,
  short reports, program documentation, invitations, \ldots
  \index{article class}
\item [\normalfont\texttt{proc}] a class for proceedings based on the article class.
  \index{proc class}
\item [\normalfont\texttt{minimal}] is as small as it can get.
It only sets a page size and a base font. It is mainly used for debugging
purposes.
  \index{minimal class}
\item [\normalfont\texttt{report}] for longer reports containing several chapters, small
  books, PhD theses, \ldots \index{report class}
\item [\normalfont\texttt{book}] for real books \index{book class}
\item [\normalfont\texttt{slides}] for slides. The class uses big sans serif
  letters. You might want to consider using the Beamer class instead.
        \index{slides class}
\end{description}
\end{lined}
\end{table}

\begin{table}[!bp]
\caption{Document Class Options.} \label{options}
\begin{lined}{\textwidth}
\begin{flushleft}
\begin{description}
\item[\normalfont\texttt{10pt}, \texttt{11pt}, \texttt{12pt}] \quad Sets the size
  of the main font in the document. If no option is specified,
  \texttt{10pt} is assumed.  \index{document font size}\index{base
    font size}
\item[\normalfont\texttt{a4paper}, \texttt{letterpaper}, \ldots] \quad Defines
  the paper size. The default size is \texttt{letterpaper}. Besides
  that, \texttt{a5paper}, \texttt{b5paper}, \texttt{executivepaper},
  and \texttt{legalpaper} can be specified.  \index{legal paper}
  \index{paper size}\index{A4 paper}\index{letter paper} \index{A5
    paper}\index{B5 paper}\index{executive paper}

\item[\normalfont\texttt{fleqn}] \quad Typesets displayed formulae left-aligned
  instead of centred.

\item[\normalfont\texttt{leqno}] \quad Places the numbering of formulae on the
  left hand side instead of the right.

\item[\normalfont\texttt{titlepage}, \texttt{notitlepage}] \quad Specifies
  whether a new page should be started after the \wi{document title}
  or not. The \texttt{article} class does not start a new page by
  default, while \texttt{report} and \texttt{book} do.  \index{title}

\item[\normalfont\texttt{onecolumn}, \texttt{twocolumn}] \quad Instructs \LaTeX{} to typeset the
  document in \wi{one column} or \wi{two column}s.

\item[\normalfont\texttt{twoside, oneside}] \quad Specifies whether double or
  single sided output should be generated. The classes
  \texttt{article} and \texttt{report} are \wi{single sided} and the
  \texttt{book} class is \wi{double sided} by default. Note that this
  option concerns the style of the document only. The option
  \texttt{twoside} does \emph{not} tell the printer you use that it
  should actually make a two-sided printout.
\item[\normalfont\texttt{landscape}] \quad Changes the layout of the document to print in landscape mode.
\item[\normalfont\texttt{openright, openany}] \quad Makes chapters begin either
  only on right hand pages or on the next page available. This does
  not work with the \texttt{article} class, as it does not know about
  chapters. The \texttt{report} class by default starts chapters on
  the next page available and the \texttt{book} class starts them on
  right hand pages.

\end{description}
\end{flushleft}
\end{lined}
\end{table}

Example: An input file for a \LaTeX{} document could start with the
line
\begin{code}
\ci{documentclass}\verb|[11pt,twoside,a4paper]{article}|
\end{code}
which instructs \LaTeX{} to typeset the document as an \emph{article}
with a base font size of \emph{eleven points}, and to produce a
layout suitable for \emph{double sided} printing on \emph{A4 paper}.
\pagebreak[2]

\subsection{Packages}
\index{package} While writing your document, you will probably find
that there are some areas where basic \LaTeX{} cannot solve your
problem. If you want to include \wi{graphics}, \wi{coloured text} or
source code from a file into your document, you need to enhance the
capabilities of \LaTeX.  Such enhancements are called packages.
Packages are activated with the
\begin{lscommand}
\ci{usepackage}\verb|[|\emph{options}\verb|]{|\emph{package}\verb|}|
\end{lscommand}
\noindent command, where \emph{package} is the name of the package and
\emph{options} is a list of keywords that trigger special features in
the package. The \ci{usepackage} command goes into the preamble of the
document. See section \ref{sec:structure} for details.

Some packages come with the \LaTeXe{} base distribution
(See Table~\ref{packages}). Others are provided separately. You may
find more information on the packages installed at your site in your
\guide. The prime source for information about \LaTeX{} packages is \companion.
It contains descriptions on hundreds of packages, along with
information of how to write your own extensions to \LaTeXe.

Modern \TeX{} distributions come with a large number of packages
preinstalled. If you are working on a Unix system, use the command
\texttt{texdoc} for accessing package documentation.


\begin{table}[btp]
\caption{Some of the Packages Distributed with \LaTeX.} \label{packages}
\begin{lined}{\textwidth}
\begin{description}
\item[\normalfont\pai{doc}] Allows the documentation of \LaTeX{} programs.\\
 Described in \texttt{doc.dtx}\footnote{This file should be installed
   on your system, and you should be able to get a \texttt{dvi} file
   by typing \texttt{latex doc.dtx} in any directory where you have
   write permission. The same is true for all the
   other files mentioned in this table.}  and in \companion.

\item[\normalfont\pai{exscale}] Provides scaled versions of the
  math extension  font.\\
  Described in \texttt{ltexscale.dtx}.

\item[\normalfont\pai{fontenc}] Specifies which \wi{font encoding}
  \LaTeX{} should use.\\
  Described in \texttt{ltoutenc.dtx}.

\item[\normalfont\pai{ifthen}] Provides commands of the form\\
  `if\ldots then do\ldots otherwise do\ldots.'\\ Described in
  \texttt{ifthen.dtx} and \companion.

\item[\normalfont\pai{latexsym}] To access the \LaTeX{} symbol
  font, you should use the \texttt{latexsym} package. Described in
  \texttt{latexsym.dtx} and in \companion.

\item[\normalfont\pai{makeidx}] Provides commands for producing
  indexes.  Described in section~\ref{sec:indexing} and in \companion.

\item[\normalfont\pai{syntonly}] Processes a document without
  typesetting it.

\item[\normalfont\pai{inputenc}] Allows the specification of an
  input encoding such as ASCII, ISO Latin-1, ISO Latin-2, 437/850 IBM
  code pages,  Apple Macintosh, Next, ANSI-Windows or user-defined one.
  Described in \texttt{inputenc.dtx}.
\end{description}
\end{lined}
\end{table}


\subsection{Page Styles}

\LaTeX{} supports three predefined \wi{header}/\wi{footer}
combinations---so-called \wi{page style}s. The \emph{style} parameter
of the \index{page style!plain@\texttt{plain}}\index{plain@\texttt{plain}}
\index{page style!headings@\texttt{headings}}\index{headings@texttt{headings}}
\index{page style!empty@\texttt{empty}}\index{empty@\texttt{empty}}
\begin{lscommand}
\ci{pagestyle}\verb|{|\emph{style}\verb|}|
\end{lscommand}
\noindent command defines which one to use.
Table~\ref{pagestyle}
lists the predefined page styles.

\begin{table}[!hbp]
\caption{The Predefined Page Styles of \LaTeX.} \label{pagestyle}
\begin{lined}{\textwidth}
\begin{description}

\item[\normalfont\texttt{plain}] prints the page numbers on the bottom
  of the page, in the middle of the footer. This is the default page
  style.

\item[\normalfont\texttt{headings}] prints the current chapter heading
  and the page number in the header on each page, while the footer
  remains empty.  (This is the style used in this document)
\item[\normalfont\texttt{empty}] sets both the header and the footer
  to be empty.

\end{description}
\end{lined}
\end{table}

It is possible to change the page style of the current page
with the command
\begin{lscommand}
\ci{thispagestyle}\verb|{|\emph{style}\verb|}|
\end{lscommand}
A description how to create your own
headers and footers can be found in \companion{} and in section~\ref{sec:fancy} on page~\pageref{sec:fancy}.
%
% Pointer to the Fancy headings Package description !
%

\section{Files You Might Encounter}

When you work with \LaTeX{} you will soon find yourself in a maze of
files with various \wi{extension}s and probably no clue. The following
list explains the various \wi{file types} you might encounter when
working with \TeX{}. Please note that this table does not claim to be
a complete list of extensions, but if you find one missing that you
think is important, please drop me a line.

\begin{description}

\item[\eei{.tex}] \LaTeX{} or \TeX{} input file. Can be compiled with
  \texttt{latex}.
\item[\eei{.sty}] \LaTeX{} Macro package. Load this
  into your \LaTeX{} document using the \ci{usepackage} command.
\item[\eei{.dtx}] Documented \TeX{}. This is the main distribution
  format for \LaTeX{} style files. If you process a .dtx file you get
  documented macro code of the \LaTeX{} package contained in the .dtx
  file.
\item[\eei{.ins}] The installer for the files contained in the
  matching .dtx file. If you download a \LaTeX{} package from the net,
  you will normally get a .dtx and a .ins file. Run \LaTeX{} on the
  .ins file to unpack the .dtx file.
\item[\eei{.cls}] Class files define what your document looks
  like. They are selected with the \ci{documentclass} command.
\item[\eei{.fd}] Font description file telling  \LaTeX{} about new fonts.
\end{description}
The following files are generated when you run \LaTeX{} on your input
file:

\begin{description}
\item[\eei{.dvi}] Device Independent File. This is the main result of a classical \LaTeX{}
  compile run. Look at its content with a DVI previewer
  program or send it to a printer with \texttt{dvips} or a
  similar application. If you are using \hologo{pdfLaTeX} then you should not see any of these files.
\item[\eei{.log}] Gives a detailed account of what happened during the
  last compiler run.
\item[\eei{.toc}] Stores all your section headers. It gets read in for the
  next compiler run and is used to produce the table of contents.
\item[\eei{.lof}] This is like .toc but for the list of figures.
\item[\eei{.lot}] And again the same for the list of tables.
\item[\eei{.aux}] Another file that transports information from one
  compiler run to the next. Among other things, the .aux file is used
  to store information associated with cross-references.
\item[\eei{.idx}] If your document contains an index. \LaTeX{} stores all
  the words that go into the index in this file. Process this file with
  \texttt{makeindex}. Refer to section \ref{sec:indexing} on
  page \pageref{sec:indexing} for more information on indexing.
\item[\eei{.ind}] The processed .idx file, ready for inclusion into your
  document on the next compile cycle.
\item[\eei{.ilg}] Logfile telling what \texttt{makeindex} did.
\end{description}


% Package Info pointer
%
%



%
% Add Info on page-numbering, ...
% \pagenumbering

\section{Big Projects}
When working on big documents, you might want to split the input file
into several parts. \LaTeX{} has two commands that help you to do
that.

\begin{lscommand}
\ci{include}\verb|{|\emph{filename}\verb|}|
\end{lscommand}
\noindent Use this command in the document body to insert the
contents of another file named \emph{filename.tex}. Note that \LaTeX{}
will start a new page
before processing the material input from \emph{filename.tex}.

The second command can be used in the preamble. It allows you to
instruct \LaTeX{} to only input some of the \verb|\include|d files.
\begin{lscommand}
\ci{includeonly}\verb|{|\emph{filename}\verb|,|\emph{filename}%
\verb|,|\ldots\verb|}|
\end{lscommand}
After this command is executed in the preamble of the document, only
\ci{include} commands for the filenames that are listed in the
argument of the \ci{includeonly} command will be executed.

The \ci{include} command starts typesetting the included text on a new
page. This is helpful when you use \ci{includeonly}, because the
page breaks will not move, even when some include files are omitted.
Sometimes this might not be desirable. In this case, use the
\begin{lscommand}
\ci{input}\verb|{|\emph{filename}\verb|}|
\end{lscommand}
\noindent command. It simply includes the file specified.
No flashy suits, no strings attached.


To make \LaTeX{} quickly check your document use the \pai{syntonly}
package. This makes \LaTeX{} skim through your document only checking for
proper syntax and usage of the commands, but doesn't produce any (pdf) output.
As \LaTeX{} runs faster in this mode you may save yourself valuable time.
Usage is very simple:

\begin{verbatim}
\usepackage{syntonly}
\syntaxonly
\end{verbatim}
When you want to produce pages, just comment out the second line
(by adding a percent sign).


%

% Local Variables:
% TeX-master: "lshort2e"
% mode: latex
% mode: flyspell
% End:
