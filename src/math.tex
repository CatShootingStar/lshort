% The Not So Short Introduction to LaTeX
%
% Copyright (C) 1995--2022 Tobias Oetiker, Marcin Serwin, Hubert Partl,
% Irene Hyna, Elisabeth Schlegl and Contributors.
%
% This document is free software: you can redistribute it and/or modify it
% under the terms of the GNU General Public License as published by the Free
% Software Foundation, either version 3 of the License, or (at your option) any
% later version.
%
% This document is distributed in the hope that it will be useful, but WITHOUT
% ANY WARRANTY; without even the implied warranty of MERCHANTABILITY or FITNESS
% FOR A PARTICULAR PURPOSE.  See the GNU General Public License for more
% details.
%
% You should have received a copy of the GNU General Public License along with
% this document.  If not, see <https://www.gnu.org/licenses/>.

% !TEX root = ./lshort.tex
%%%%%%%%%%%%%%%%%%%%%%%%%%%%%%%%%%%%%%%%%%%%%%%%%%%%%%%%%%%%%%%%
% Contents: Math typesetting with LaTeX
% $Id$
%
% Changes by Stefan M. Moser: 2008/10/22
%
% -Section 2: "Single Equations": added comment about preference of
%  equation* over \[
% -Replaced (almost) all examples with \[ by equation*
% -New section 4: "Single Equations that are Too Long: multline"
% -New section 5: "Multiple Equations"
% -Section 6: "Arrays and Matrices": made a full section and added
%  some material
% -Section 9: "Theorems, Lemmas, ...": added a subsection about proofs
%  with new material
%
% Other Changes:
% -in lshort.sty: 
%    *example environment adapted: changed in three places
%     \textwidth by \linewidth. This is necessary for
%     example-environment within a itemize-list.
%    *added \RequirePackage[retainorgcmds]{IEEEtrantools}
%
%%%%%%%%%%%%%%%%%%%%%%%%%%%%%%%%%%%%%%%%%%%%%%%%%%%%%%%%%%%%%%%%%

\chapter{Typesetting Mathematical Formulae}\label{chap:math}

\begin{intro}
  Now you are ready! In this chapter, we will attack the main strength
  of \TeX{}: mathematical typesetting. But be warned, this chapter
  only scratches the surface. While the things explained here are
  sufficient for many people, don't despair if you can't find a
  solution to your mathematical typesetting needs here. It is highly likely
  that your problem is addressed in \hologo{AmSLaTeX}.
\end{intro}

\section{The \hologo{AmSLaTeX} bundle}

If you want to typeset (advanced) \wi{mathematics}, you should
use \hologo{AmSLaTeX}. The \hologo{AmSLaTeX} bundle is a collection of packages and classes for
mathematical typesetting. We will mostly deal with the \pai{amsmath} package
which is a part of the bundle. \hologo{AmSLaTeX} is produced by The \emph{\wi{American Mathematical Society}}
and it is used extensively for mathematical typesetting. \LaTeX{} itself does provide
some basic features and environments for mathematics, but they are limited (or
maybe it's the other way around: \hologo{AmSLaTeX} is \emph{unlimited}!) and
in some cases inconsistent.

\hologo{AmSLaTeX} is a part of the required distribution and is provided
with all recent \LaTeX{} distributions. In this chapter, we assume
\pai*{amsmath} is loaded in the preamble; \mintinline{latex}|\usepackage{amsmath}|.

\section{Single Equations}

A mathematical formula can be typeset in-line within a paragraph (\emph{\wi{text style}}), or the paragraph can be broken and the formula typeset separately
(\emph{\wi{display style}}). Mathematical \wi{equation}s
\emph{within} a paragraph are entered
between \ai{\$} and \ai{\$}:
\begin{example}
Add $a$ squared and $b$ squared
to get $c$ squared. Or, using 
a more mathematical approach:
$a^2 + b^2 = c^2$
\end{example}
\begin{example}
\TeX{} is pronounced as 
$\tau\epsilon\chi$\\[5pt]
100~m$^{3}$ of water\\[5pt]
This comes from my $\heartsuit$
\end{example}

If you want your larger equations to be set apart
from the rest of the paragraph, it is preferable to \emph{display} them
rather than to break the paragraph apart.
To do this, you enclose them between \verb|\begin{|\ei{equation}\verb|}| and
\verb|\end{equation}|.\footnote{This is an \pai{amsmath} environment. If you don't
  have access to the package for some obscure reason, you can use \LaTeX's own
  \ei{displaymath} environment instead.} You can then \csi{label} an equation number and refer to
it somewhere else in the text by using the \csi{eqref} command. If you want to
name the equation something specific, you \csi{tag} it instead.
\begin{example}
Add $a$ squared and $b$ squared
to get $c$ squared. Or, using
a more mathematical approach
 \begin{equation}
   a^2 + b^2 = c^2
 \end{equation}
Einstein says
 \begin{equation}
   E = mc^2 \label{clever}
 \end{equation}
He didn't say
 \begin{equation}
  1 + 1 = 3 \tag{dumb}
 \end{equation}
This is a reference to%
~\eqref{clever}. 
\end{example}

If you don't want \LaTeX{} to number the equations, use the starred
version of \ei{equation} using an asterisk, \ei{equation*}, or even easier, enclose the
equation in \csi{[} and \csi{]}:\footnote{\index{equation!amsmath@\textsf{amsmath}}\index{equation!LaTeX@\LaTeX{}}
  This is again from \pai{amsmath}.
  Standard \LaTeX{}'s has only the \ei{equation} environment without the star.}
\begin{example}
Add $a$ squared and $b$ squared
to get $c$ squared. Or, using
a more mathematical approach
 \begin{equation*}
   a^2 + b^2 = c^2
 \end{equation*}
or you can type less for the
same effect:
 \[ a^2 + b^2 = c^2 \]
\end{example}
While \csi{[} is short and sweet, it does not allow switching between numbered and not numbered style as easily as
\ei{equation} and \ei{equation*}.

Note the difference in typesetting style between \wi{text style} and \wi{display style}
equations:
\begin{example}
This is text style: 
$\lim_{n \to \infty} 
 \sum_{k=1}^n \frac{1}{k^2} 
 = \frac{\pi^2}{6}$.
And this is display style:
 \begin{equation}
  \lim_{n \to \infty} 
  \sum_{k=1}^n \frac{1}{k^2} 
  = \frac{\pi^2}{6}
 \end{equation}
\end{example}

In text style, you may enclose tall or deep math expressions or sub
expressions in \csi{smash}. This makes \LaTeX{} ignore the height of
these expressions. This keeps the line spacing even, but risks overlapping the
text around them.

\begin{example}
A $d_{e_{e_p}}$ mathematical
expression  followed by a
$h^{i^{g^h}}$ expression. As
opposed to a smashed 
\smash{$d_{e_{e_p}}$}
expression 
followed by a
\smash{$h^{i^{g^h}}$}
expression.
\end{example}

\subsection{Math Mode}

There are also differences between \emph{\wi{math mode}} and \emph{text mode}. For
example, in \emph{math mode}:

\begin{enumerate}

  \item\index{spacing!math mode} Most spaces and line breaks do not have any significance, as all spaces
  are either derived logically from the mathematical expressions, or
  have to be specified with special commands such as \csi{,}, \csi{quad} or
  \csi{qquad} (we'll get back to that later, see \autoref{sec:math-spacing}).

  \item Empty lines are not allowed. Only one paragraph per formula.

  \item Each letter is considered to be the name of a variable and will be
        typeset as such. If you want to typeset normal text within a formula
        (normal upright font and normal spacing) then you have to enter the
        text using the \csi{text} command (see also \autoref{sec:fontsz}~on
        \autopageref{sec:fontsz}).

\end{enumerate}
\begin{example}
$\forall x \in \mathbf{R}
  \colon x^{2} \geq 0$
\end{example}
\begin{example}
$x^{2} \geq 0 \text{ for all }
  x\in\mathbf{R}$
\end{example}

Mathematicians can be very fussy about which symbols are used:
it would be conventional here to use the `\wi{blackboard bold}' font,%
\index{bold symbols} which is obtained using \csi{mathbb} from the
package \pai*{amssymb}.\footnote{\pai{amssymb} is not a part
  of the \hologo{AmSLaTeX} bundle, but it is perhaps still a part of your \LaTeX{}
  distribution.}
The last example becomes
\begin{example}
$x^{2} \geq 0
 \text{ for all } x 
 \in \mathbb{R}$
\end{example}
See \autoref{mathalpha} on \autopageref{mathalpha} and
\autoref{mathfonts} on \autopageref{mathfonts} for more math fonts.

\section{Building Blocks of a Mathematical Formula}

In this section, we describe the most important commands used in mathematical
typesetting. Most of the commands in this section will not require
\textsf{amsmath} (if they do, it will be stated clearly), but load it anyway.

\emph{Lowercase \wi{Greek letters}} are entered as \csi{alpha},
\csi{beta}, \csi{gamma}, \ldots, uppercase letters
are entered as \csi{Gamma}, \csi{Delta}, \ldots\footnote{There is no
  uppercase Alpha, Beta etc.\ defined in \LaTeX{} because it looks the same as a
  normal roman A, B\ldots{}}

Take a look at \autoref{greekletters} on \autopageref{greekletters} for a
list of Greek letters.
\begin{example}
$\lambda,\xi,\pi,\theta,
 \mu,\Phi,\Omega,\Delta$
\end{example}

\emph{Exponents, Superscripts and Subscripts} can be specified using%
\index{exponent}\index{subscript}\index{superscript}
the \ai{\^} and the \ai{\_} characters.
Most math mode commands act only on the next character, so if you
want a command to affect several characters, you have to group them
together using curly braces: \verb|{...}|.

\autoref{binaryrel} on \autopageref{binaryrel} lists a lot of binary
relations like $\subseteq$ and $\perp$.

\begin{example}
$p^3_{ij}$ 
\qquad%!hide
$m_\text{Knuth}$
\qquad%!hide
$\sum_{k=1}^3 k$ \\[5pt]
\qquad%!hide
$a^x+y \neq a^{x+y}$
\qquad%!hide
$e^{x^2} \neq {(e^x)}^2$
\end{example}

The \emph{\wi{square root}} is entered as \csi{sqrt}; the
\carg{n}-th root is generated with \csi{sqrt}[n:o]. The size of
the root sign is determined automatically by \LaTeX. If just the sign
is needed, use \csi{surd}.

\begin{example}
$\sqrt{x} = x^{1/2}$
\qquad%!hide
$\sqrt[3]{2}$ \\
$\sqrt{x^{2} + \sqrt{y}}$
\qquad%!hide
$\surd[x^2 + y^2]$
\end{example}

While the \emph{\wi{dot}} sign to indicate%
\index{dots!three}\index{vertical!dots}\index{horizontal!dots}
the multiplication operation is normally left out, it is sometimes written
to help the eye in grouping a formula.
Use \csi{cdot} to typeset a single centred dot. \csi{cdots} is
three centred \emph{\wi{dots}} while \csi{ldots} sets the dots low (on the
baseline). Besides that, there are \csi{vdots} for
vertical and \csi{ddots} for \wi{diagonal dots}. There are more examples in
\autoref{sec:arraymat}.
\begin{example}
$\Psi = v_1 \cdot v_2
 \cdot \ldots$ \\
$n! = 1 \cdot 2 
 \cdots (n-1) \cdot n$
\end{example}

The commands \csi{overline} and \csi{underline} create
\emph{horizontal lines} directly over or under an expression:%
\index{horizontal!line}\index{line!horizontal}
\begin{example}
$0.\overline{3} = 
 \underline{\underline{1/3}}$
\end{example}

The commands \csi{overbrace} and \csi{underbrace} create
long \emph{horizontal braces} over or under an expression:%
\index{horizontal!brace}\index{brace!horizontal}
\begin{example}
$\underbrace{
  \overbrace{(a+b+c)}^6 
 \cdot \overbrace{(d+e+f)}^7}
 _\text{meaning of life} = 42$
\end{example}

To add mathematical accents\index{mathematical!accents} such as \emph{small
  arrows} or \emph{\wi{tilde}} signs to variables, the commands
given in \autoref{mathacc} on \autopageref{mathacc} might be useful.  Wide hats and
tildes covering several characters are generated with \csi{widetilde}
and \csi{widehat}. Notice the difference between \csi{hat} and \csi{widehat} and the placement of
\csi{bar} for a variable with subscript. The \wi{apostrophe} mark
\ai{'} gives a \wi{prime}:
% a dash is --
\begin{example}
$f(x) = x^2$ \\
$f'(x) = 2x$ \\
$f''(x) = 2$ \\
$\hat{XY} \neq \widehat{XY}$ \\
$\bar{x_0} \neq \bar{x}_0$
\end{example}

\emph{Vectors}\index{vectors} are often specified by adding small
\wi{arrow symbols} on the tops of variables. This is done with the
\csi{vec} command. The two commands \csi{overrightarrow} and
\csi{overleftarrow} are useful to denote the vector from $A$ to $B$:
\begin{example}
$\vec{a}$
\quad%!hide
$\vec{AB}$
\quad%!hide
$\overrightarrow{AB}$
\end{example}

Names of functions are often typeset in an upright
font, and not in italics as variables are, so \LaTeX{} supplies the
following commands to typeset the most common function names:%
\index{mathematical!functions}

\begin{center}
  \begin{tabular}{llllll}
    \csi{arccos} & \csi{cos}  & \csi{csc} & \csi{exp}  & \csi{ker}    & \csi{limsup} \\
    \csi{arcsin} & \csi{cosh} & \csi{deg} & \csi{gcd}  & \csi{lg}     & \csi{ln}     \\
    \csi{arctan} & \csi{cot}  & \csi{det} & \csi{hom}  & \csi{lim}    & \csi{log}    \\
    \csi{arg}    & \csi{coth} & \csi{dim} & \csi{inf}  & \csi{liminf} & \csi{max}    \\
    \csi{sinh}   & \csi{sup}  & \csi{tan} & \csi{tanh} & \csi{min}    & \csi{Pr}     \\
    \csi{sec}    & \csi{sin}                                                         \\
  \end{tabular}
\end{center}

\begin{example}
\begin{equation*}
  \lim_{x \rightarrow 0}
  \frac{\sin x}{x}=1
\end{equation*}
\end{example}

For functions missing from the list, use the \csi{DeclareMathOperator}
command. There is even a starred version for functions with limits.
This command works only in the preamble.

\begin{example}
%!showbegin !hide
% In preamble
\DeclareMathOperator{\argh}{argh}
\DeclareMathOperator*{\nut}{Nut}
% ...
%!showend !hide
\begin{equation*}
  3\argh = 2\nut_{x=1}    
\end{equation*}
\end{example}

For the \wi{modulo function}, there are two commands: \csi{bmod} for the
binary operator ``$a \bmod b$'' and \csi{pmod}
for expressions
such as ``$x\equiv a \pmod{b}$:''
\begin{example}
$a\bmod b \\
 x\equiv a \pmod{b}$
\end{example}

A built-up \emph{\wi{fraction}} is typeset with the
\begin{lscommand}
  \csi{frac}[numerator:m, denominator:m]%chktex 26
\end{lscommand}
command.
\begin{example}
\ldots{} And since
\begin{equation*}
  \frac{1}{2} 
  + \frac{1}{3}
  = \frac{5}{6},
\end{equation*}
then the answers is
$\frac{5}{6}$.
\end{example}

In in-line equations, the fraction is shrunk to
fit the line. This style is obtainable in display style with \csi{tfrac}. The
reverse, i.e.\ display style fraction in text, is made with \csi{dfrac}.
Often the slashed form $1/2$ is preferable, because it looks better
for small amounts of `fraction material:'
\begin{example}
In display style:
\begin{equation*}
  3/8
  \qquad%!hide
  \frac{3}{8} 
  \qquad%!hide
  \tfrac{3}{8}
\end{equation*}
\end{example}

\begin{example}
In text style: \\
$1\frac{1}{2}$~hours
\qquad%!hide
$1\dfrac{1}{2}$~hours
\end{example}

Here the \csi{partial} command for \wi{partial derivative}s is used:
\begin{example}
\begin{equation*} 
  \sqrt{\frac{x^2}{k+1}}
  \qquad%!hide
  x^\frac{2}{k+1}
  \qquad%!hide
  \frac{\partial^2f}
  {\partial x^2} 
\end{equation*}
\end{example}

To typeset \wi{binomial coefficient}s or similar structures, use
the command \csi{binom} from \pai{amsmath}:
\begin{example}
Pascal's rule is
\begin{equation*}
 \binom{n}{k} =\binom{n-1}{k}
 + \binom{n-1}{k-1}
\end{equation*}
\end{example}

For \wi{binary relations} it may be useful to stack symbols over each other.
\csi{stackrel}[above:m, below:m] puts the symbol given%chktex 26
in \carg{above} in superscript-like size over \carg{below} which
is set in its usual position.
\begin{example}
\begin{equation*}
 f_n(x) \stackrel{*}{\approx} 1
\end{equation*}
\end{example}

The \emph{\wi{integral operator}} is generated with \csi{int}, the
\emph{\wi{sum operator}} with \csi{sum}, and the \emph{\wi{product operator}}
with \csi{prod}. The upper and lower limits are specified with~\ai{\^{}}
and~\ai{\_} like superscripts and subscripts:
\begin{example}
\begin{equation*}
\sum_{i=1}^n
  \qquad%!hide
  \int_0^{\frac{\pi}{2}}
  \qquad%!hide
  \prod_\epsilon
\end{equation*}
\end{example}

To get more control over the placement of indices in complex
expressions, \pai{amsmath} provides the \csi{substack} command:
\begin{example}
\begin{equation*}
\sum^n_{\substack{0<i<n \\ 
        j\subseteq i}}
   P(i,j) = Q(i,j)
\end{equation*}
\end{example}

\LaTeX{} provides all sorts of symbols for \emph{\wi{bracketing}} and other
\emph{\wi{delimiters}} (e.g.~$[\;\langle\;\|\;\updownarrow$). %chktex 9
Round and square brackets can be entered with the corresponding keys and
curly braces with \csi{\{}, but all other delimiters are generated with
special commands (e.g.~\csi{updownarrow}).
\begin{example}
\begin{equation*}
{a,b,c} \neq \{a,b,c\}
\end{equation*}
\end{example}

If you put \csi{left} in front of an opening delimiter and
\csi{right} in front of a closing delimiter, \LaTeX{} will automatically
determine the correct size of the delimiter. Note that you must close
every \csi{left} with a corresponding \csi{right}. If you
don't want anything on the right, use the invisible \csi{right.} (with a dot):
\begin{example}
\begin{equation*}
1 + \left(\frac{1}{1-x^{2}}
    \right)^3 \qquad 
\left. \ddagger
  \frac{~}{~}\right)
\end{equation*}
\end{example}

In some cases it is necessary to specify the correct size of a
mathematical delimiter\index{mathematical!delimiter} by hand,
which can be done using the commands \csi{big}, \csi{Big}, \csi{bigg} and
\csi{Bigg} as prefixes to most delimiter commands:
\begin{chktexignore}
  \begin{example}
$\Big((x+1)(x-1)\Big)^{2}$\\
$\big( \Big( \bigg( \Bigg($ 
$\big\} \Big\} \bigg\} \Bigg\}$ 
$\big\| \Big\| \bigg\| \Bigg\|$ 
$\big\Downarrow \Big\Downarrow 
\bigg\Downarrow \Bigg\Downarrow$
\end{example}
\end{chktexignore}

For a list of all delimiters available, see \autoref{tab:delimiters} on \autopageref{tab:delimiters}.

You can find large list of symbols supported in math mode in
\autoref{symbols}.

\section{Single Equations that are Too Long: multline}%
\index{long equations}\label{sec:multline}

If an equation is too long, we have to wrap it somehow. Unfortunately,
wrapped equations are usually less easy to read than not wrapped
ones. To improve the readability, there are certain rules on how to do
the wrapping:
\begin{enumerate}
  \item In general one should always wrap an equation \emph{before} an
        equality sign or an operator.
  \item A wrap before an equality sign is preferable to a wrap before
        any operator.
  \item A wrap before a plus- or minus-operator is preferable to a wrap
        before a multiplication-operator.
  \item Any other type of wrap should be avoided if at all possible.
\end{enumerate}
The easiest way to achieve such a wrapping is the use of the
\ei{multline} en\-vi\-ron\-ment:\footnote{The
  \ei{multline} environment is from \pai{amsmath}.}
\begin{example}
\begin{multline}
  a + b + c + d + e + f 
  + g + h + i  
  \\
  = j + k + l + m + n 
\end{multline}
\end{example}
\noindent
The difference from the \ei{equation} environment is that an arbitrary
line-break (or also multiple line-breaks) can be introduced. This is
done by putting a \csi{\bs} on those places where the equation needs
to be wrapped. Similarly to \ei{equation*} there also exists a
\ei{multline*} version for preventing an equation number.

Often the
\ei{IEEEeqnarray} environment (see \autoref{sec:IEEEeqnarray})
will yield better results.  Consider the following
situation:
\begin{example}
\begin{equation}
  a = b + c + d + e + f 
  + g + h + i + j 
  + k + l + m + n + o + p  
  \label{eq:equation_too_long}
\end{equation}
\end{example}
\noindent
Here it is actually the RHS that is too long to fit on one line. The
\ei{multline} environment creates the following output:
\begin{example}
\begin{multline}
  a = b + c + d + e + f 
  + g + h + i + j \\
  + k + l + m + n + o + p
\end{multline}
\end{example}

This is better than~\eqref{eq:equation_too_long}, but
it has the disadvantage that the equality sign loses its natural
greater importance with respect to the plus operator in front of
$k$. The better solution is provided by the
\ei{IEEEeqnarray} environment that will be discussed in detail in
\autoref{sec:IEEEeqnarray}.

\section{Multiple Equations}%
\index{equation!multiple}\label{sec:IEEEeqnarray}

In the most general situation we have a sequence of several
equalities that do not fit onto one line. Here we need to work with
vertical alignment in order to keep the array of equations in a nice
and readable structure.

Before we offer our suggestions on how to do this, we start with a few
bad examples that show the biggest drawbacks of some common solutions.

\subsection{Problems with Traditional Commands}%
\label{sec:problems_traditional}

To group multiple equations the
\ei{align} environment\footnote{The \ei{align} environment can
  also be used to group several blocks of equations beside each other.
  Another excellent use case for the
  \ei{IEEEeqnarray} environment. Try an argument like
  \cargv{\{rCl+rCl\}}.} could be used:
\begin{example}
\begin{align}
  a & = b + c \\
  & = d + e
\end{align}
\end{example}
this approach fails once a single line is too long:
\begin{example}
\begin{align}
  a & = b + c \\
  & = d + e + f
  + g + h + i \nonumber \\
  & + j + k + l \\
  & = m + n + o + p
\end{align}
\end{example}
Here $+\:m$ should be below $d$ and not below the equality sign. A
\TeX{}pert will point out that \mintinline{latex}|\mathrel{\phantom{=}} \negmedspace{}|,
would add the necessary space in front of \verb|+m+n+o|, but since most users
lack that kind of imagination, a simpler solution would be nice.

This is the moment where the \ei{eqnarray} environment bursts onto the scene:
\begin{example}
\begin{eqnarray}
  a & = & b + c \\
  & = & d + e + f
  + g + h + i \nonumber \\
  && +\: j + k + l \\
  & = & m + n + o + p
\end{eqnarray}
\end{example}

It is better but still not optimal. The spaces around the equality signs are too big.
Particularly, they are \emph{not} the same as in the
\ei{multline} and \ei{equation} environments:
\begin{example}
\begin{eqnarray}
  a & = & a = a
\end{eqnarray}
\end{example}

\noindent \ldots and the expression sometimes overlaps with the equation number even
though there would be enough room on the left:
\begin{example}
\begin{eqnarray}
  a & = & b + c 
  \\
  & = & d + e + f + g + h^2 
  + i^2 + j 
  \label{eq:faultyeqnarray}
\end{eqnarray}
\end{example}

\noindent While the environment offers a command \csi{lefteqn} that can
be used when the LHS is too long:
\begin{example}
\begin{eqnarray}
  \lefteqn{a + b + c + d 
    + e + f + g + h}\nonumber\\
  & = & i + j + k + l + m 
  \\
  & = & n + o + p + q + r + s
\end{eqnarray}
\end{example}
\noindent This is not optimal either as the RHS is too short and the array is
not properly centred:
\begin{example}
\begin{eqnarray}
  \lefteqn{a + b + c + d 
    + e + f + g + h} 
  \nonumber \\
  & = & i + j 
\end{eqnarray}
\end{example}

\noindent Having badmouthed the competition sufficiently, I can now steer you gently towards the glorious \ldots

\subsection{IEEEeqnarray Environment}%
\label{sec:IEEEeqnarray_intro}

The \ei{IEEEeqnarray} environment is a very powerful command with
many options. Here, we will only introduce its basic
functionalities. For more information please refer to the
manual.\footnote{For the official manual see~\cite{IEEEtran_HOWTO}.
  The part about \texttt{IEEEeqnarray}
  can be found in Appendix~F.}

First of all, in order to be able to use the
\ei{IEEEeqnarray} environment one needs to load the
package\footnote{The \pai{IEEEtrantools} package may not be included in your setup, it can be found on CTAN.}
\pai*{IEEEtrantools}. Include the following line in the header of
your document: \small
\begin{minted}{latex}
\usepackage{IEEEtrantools}
\end{minted}
\normalsize

The strength of \ei{IEEEeqnarray} is the ability to specify
the number of \emph{columns} in the equation array. Usually, this
specification will be \cargv{\{rCl\}}, i.e., three columns, the
first column right-justified, the middle one centred with a little
more space around it (therefore we specify capital \cargv{C} instead of
lower-case \cargv{c}) and the third column left-justified:
\begin{example}
\begin{IEEEeqnarray}{rCl}
  a & = & b + c 
  \\
  & = & d + e + f + g + h 
  + i + j + k \nonumber\\
  && \negmedspace{} + l 
  + m + n + o 
  \\
  & = & p + q + r + s
\end{IEEEeqnarray}
\end{example}
Any number of columns can be specified:
\verb+{c}+ will give only one column with all entries centred, or
\verb+{rCll}+ would add a fourth, left-justified column to use
for comments. Moreover, beside \cargv{l}, \cargv{c}, \cargv{r}, \cargv{L},
\cargv{C}, \cargv{R} for math mode entries there are also \cargv{s},
\cargv{t}, \cargv{u} for left, centred, and right text mode entries.
Additional space can be added with \cargv{.} and
\cargv{/} and \cargv{?} in increasing order.\footnote{For more spacing
  types refer to \autoref{sec:putting-qed-right}.}
Note the spaces around the equality signs in contrast to the space produced
by the \ei{eqnarray} environment.

\subsection{Common Usage}%
\label{sec:common-usage}

In the following we will describe how we use \ei{IEEEeqnarray} to
solve the most common problems.

If a line overlaps with the equation number as in%
~\eqref{eq:faultyeqnarray}, the command
\small
\begin{minted}{latex}
\IEEEeqnarraynumspace
\end{minted}
\normalsize
can be used: it has to be added in the corresponding line and makes
sure that the whole equation array is shifted by the size of the
equation numbers (the shift depends on the size of the number!):
instead of
\begin{example}
\begin{IEEEeqnarray}{rCl}
  a & = & b + c 
  \\
  & = & d + e + f + g + h + i
  \\
  & = & j + k + l + m + n.
\end{IEEEeqnarray}
\end{example}
we get
\begin{example}
\begin{IEEEeqnarray}{rCl}
  a & = & b + c 
  \\
  & = & d + e + f + g + h + i
  \IEEEeqnarraynumspace\\
  & = & j + k + l + m + n.
\end{IEEEeqnarray}
\end{example}

If the LHS is too long, as a replacement for the faulty
\csi{lefteqn} command, \ei{IEEEeqnarray} offers the
\csi{IEEEeqnarraymulticol} command which works in all situations:
\begin{example}
\begin{IEEEeqnarray}{rCl}
  \IEEEeqnarraymulticol{3}{l}{
    a + b + c + d + e + f 
    + g + h
  }\nonumber\\ \quad
  & = & i + j 
  \\
  & = & k + l + m
\end{IEEEeqnarray}
\end{example}
The usage is identical to the \csi{multicolumns} command in the
\ei{tabular} environment. The first argument \cargv{\{3\}}
specifies that three columns shall be combined into one which will be
left-justified \carg{\{l\}}.

Note that by inserting \csi{quad} commands one can easily adapt
the depth of the equation signs,\footnote{I think that one quad is the
  distance that looks good for most cases.} e.g.,
\begin{example}
\begin{IEEEeqnarray}{rCl}
  \IEEEeqnarraymulticol{3}{l}{
    a + b + c + d + e + f 
    + g + h
  }\nonumber\\ \qquad\qquad
  & = & i + j
  \\
  & = & k + l + m
\end{IEEEeqnarray}
\end{example}

If an equation is split into two or more lines, \LaTeX{}
interprets the first $+$ or $-$ as a sign instead of operator.
Therefore, it is necessary to add an empty group \verb|{}| before the operator: instead of
\begin{example}
\begin{IEEEeqnarray}{rCl}
  a & = & b + c 
  \\
  & = & d + e + f + g + h 
  + i + j + k \nonumber\\
  && + l + m + n + o 
  \\
  & = & p + q + r + s
\end{IEEEeqnarray}
\end{example}
we should write
\begin{example}
\begin{IEEEeqnarray}{rCl}
  a & = & b + c 
  \\
  & = & d + e + f + g + h 
  + i + j + k \nonumber\\
  && \negmedspace{} + l 
  + m + n + o 
  \\
  & = & p + q + r + s
\end{IEEEeqnarray}
\end{example}
\noindent Note the space difference between $+$ and $l$!
The construction \verb|{} + l| forces the \verb|+|-sign to be an operator rather
than just a sign, and the unwanted ensuing space between
\verb|{}| and \verb|+| is compensated by a negative medium space
\csi{negmedspace}.

If a particular line should not have an equation number, the
number can be suppressed using \csi{nonumber} (or
\csi{IEEEnonumber}). If on such a line a label
\csi{label}[\ldots: m] is defined, then this label is passed on
to the next equation number that is not suppressed. Place the labels right before the line-break
\csi{\bs} or the next to the equation it belongs to. Apart from
improving the readability of the source code this prevents a
compilation error when a \csi{IEEEmulticol} command
follows the label-definition.

There also exists a starred version where all equation numbers are
suppressed. In this case an equation number can be made to appear
using the command \csi{IEEEyesnumber}:
\begin{example}
\begin{IEEEeqnarray*}{rCl}
  a & = & b + c \\
  & = & d + e \IEEEyesnumber\\
  & = & f + g
\end{IEEEeqnarray*}
\end{example}

Sub-numbers are also easily possible using
\csi{IEEEyessubnumber}:
\begin{example}
\begin{IEEEeqnarray}{rCl}
  a & = & b + c 
  \IEEEyessubnumber\\
  & = & d + e 
  \nonumber\\
  & = & f + g 
  \IEEEyessubnumber  
\end{IEEEeqnarray}
\end{example}

\section{Units}\label{sec:units}\index{units (SI)}\index{SI}

When dealing with real world data you will often find yourself writing units
such as \qty{10}{\kg} or \qty{25}{\coulomb\per\mole}. In technical writing it
is crucial to accurately convey values to avoid errors and ambiguity.
The \emph{International System of Units} (\emph{SI})~\cite{si}, comes with a detailed set
of typesetting rules. For example you may have noticed that in the
first sentence the space between \num{10} and \unit{\kg} is smaller than
between the words.

Fortunately thanks to the logical nature of \LaTeX{} markup and the excellent
\pai*{siunitx} package you do not need to know most of the rules. Just use the special unit-commands provided by the package and all the units in your document will be
typeset correctly.

\subsection{Pitfalls of naïvely entered units}

Naïvely, one might write units like this:
\begin{example}
The speed of light
is 299792458 m/s.
\end{example}
This may look fine initially but if your document gets longer you will probably
get into a situation where an unfortunate linebreak occurs.
\begin{example}
\ldots{} The speed of light
is 299792458 m/s. Thus we
can calculate \ldots
\end{example}
A non-breaking space will fix this problem, but the next one is already waiting---the space between numbers. Digits of
long numbers should be grouped to make them easier to read. The above example
should then written like this:
\begin{example}
The speed of light is
299\,792\,458~m/s.
\end{example}
Yet another problem arises when trying to write the units inside math
mode
\begin{example}
$10~\frac{m}{s}
  \cdot 12~s = 120~m$
\end{example}
Here we need to make sure that the units are written in roman font.
The above example should be then written as
\begin{example}
$10~\frac{\mathrm{m}}
  {\mathrm{s}} 
  \cdot 12~\mathrm{s} 
  = 120~\mathrm{m}$
\end{example}
\noindent Hopefully these examples illustrate why a better method is needed.

\subsection{Basic commands of the \pai{siunitx} package}

The basic commands of \pai{siunitx} package are
\begin{lscommand}
  \csi{num}[options: o, number: m]\\
  \csi{unit}[options: o, unit: m]\\
  \csi{qty}[options:o ! number:m ! unit:m]
\end{lscommand}
The \csi{num} command typesets a number, \csi{unit} typesets a unit, and \csi{qty}
typesets a quantity, i.e.\ a number followed by a unit. All of the commands
provide optional \carg{options} argument, which is a comma delimited key-value pair
list that can influence the output of the command. The \carg{unit} argument can
be provided in one of two styles, either \enquote{literal} or
\enquote{interpreted}.

In literal mode units are entered as string of letters
\begin{example}
\unit{kg} is a unit,
\qty{10}{m/s} is a quantity
\end{example}
Spaces in literal mode are ignored. If you want to insert a product of units,
use either dot \ai{.} or tilde \ai{\~{}}.
\begin{example}
\unit{N} is 
not \unit{kg m / s^2},
but \unit{kg.m / s^2}
or \unit{kg~m / s^2}
\end{example}

On the other hand, in interpreted mode units are entered using predefined
macros
\begin{example}
\unit{\kilo\gram} is a unit,
\qty{10}{\metre\per\second}
is a quantity
\end{example}
At first glance this may seem less convenient, but this enables the package to
recognise the logical structure of the units, which means that the formatting
can be changed on the fly.
\begin{example}[examplewidth=0.8\linewidth, vertical_mode]
Which one do you prefer:
$\unit{\newton} = \unit{
  \kilo\gram\metre\per\square\second}$,
$\unit{\newton} = \unit[per-mode = fraction]{
  \kilo\gram\metre\per\second\squared}$ or
$\unit{\N} = \unit[per-mode = symbol]{
  \kg\m\per\s\tothe{2}}$?
\end{example}
The commands \csi{gram}, \csi{kilo}, \csi{per} are rather self explanatory. As you
can see we have also used the \cargv{per-mode} option to influence the unit
style. While the command \csi{tothe} typesets arbitrary superscripts, the
command \csi{of} typesets arbitrary subscripts, known as qualifiers.
\begin{example}
\unit{\kg\of{mol}},
\unit{\bel\of{i}}
\end{example}

Note that the \csi{per} command only works on the next unit. If we want it to
apply to all following units we may use \cargv{sticky-per} option.
\begin{example}[examplewidth=0.7\linewidth, vertical_mode]
The unit of thermal conductivity is
\unit{\watt\per\metre\per\kelvin}
or equivalently
\unit[sticky-per]{\m\kg\per\s\cubed\K}. 
\end{example}

Sometimes you may want to make the units use a different colour so they figure more prominently. This can be done using \cargv{unit-color}. Options
\cargv{number-color} and \cargv{color} may be used to colour only the numeric
values or whole quantities. If you want to highlight a single unit in a
compound unit, you may use the \csi{highlight} command.
\begin{example}[examplewidth=0.8\linewidth, vertical_mode]
\qty[unit-color = red   ]{9.81}{\m\per\s\squared},
\qty[number-color = blue]{9.81}{\m\per\s\squared},
\qty[color = green      ]{9.81}{\m\per\s\squared},
\qty{9.81}{\m\per\highlight{orange}\s\squared}
\end{example}

Now let's turn our attention to numbers. The numeric values will always use dot
as a decimal separator, regardless of entered symbol. If you want to use comma
as a decimal separator, change the \cargv{output-decimal-marker} option.
\begin{example}
$\pi \approx \num{3.14159}$ \\
$\pi \approx \num{3,14159}$ \\
$\pi \approx \num[
  output-decimal-marker={,}
]{3.14159}$
\end{example}

The numbers may also be entered using exponent notation. The symbol before
exponent and base may be controlled using \cargv{exponent-product} and
\cargv{exponent-base} options
\begin{example}
$\qty{1}{\tonne} = \qty{1e6}{\g}$ \\
$\qty{1}{\gibi\byte} = \qty[
  exponent-base = 2,
]{1e30}{\byte}$ \\
$\qty{1}{\L} = \qty[
  exponent-product = \cdot,
]{1e-3}{\cubic\m}$
\end{example}

If we want all numbers to be formatted using exponent notation, regardless of
their entered form, we may use \cargv{exponent-mode} option.
\begin{example}
$2^{16} = \num{65536}$ \\
$2^{16} = \num[
  exponent-mode = scientific,
]{65536}$ \\
$2^{16} = \num[
  exponent-mode = engineering,
]{65536}$ \\
$2^{16} = \num[
  exponent-mode = fixed,
]{65.536e3}$ \\
$2^{16} = \num[
  exponent-mode = fixed,
  fixed-exponent = 5,
]{65.536e3}$
\end{example}

In some fields it may be common to write uncertainty next to numbers. This may
also be done in several ways. In order to customise the appearance of
uncertainty use the \cargv{uncertainty-mode} option.
\begin{example}
$M_\oplus = \qty{
  5.9722(6) e24}{\kg}$
  \\ %!hide
$M_\oplus = \qty{
  5.9722 +- 0.0006 e24}{\kg}$
  \\ %!hide
$M_\oplus = \qty{
  5.9722 \pm 0.0006 e24}{\kg}$
  \\ %!hide
$M_\oplus = \qty[
    uncertainty-mode = separate
  ]{5.9722(6) e24}{\kg}$
  \\ %!hide
$M_\oplus = \qty[
  uncertainty-mode = full
]{5.9722(6) e24}{\kg}$
\end{example}

By default \pai{siunitx} typesets the quantities in math mode. This may be
especially jarring if you use different font for the document text. One simple
solution is to set the \cargv{mode} option to \cargv{text}.
% TODO: Add pointer to changing math fonts section when written.
\begin{example}
\setmainfont{Source Sans Pro} %!hide
\qty{1}{\degreeCelsius} is
equal to \qty{1}{\kelvin},
but \qty[
  mode = text,
]{0}{\degreeCelsius} is \qty[
  mode = text,
]{-273.15}{\kelvin}.
Bizarre, isn't it?
\end{example}

The behaviour of the commands may be customised even further. For a full list
of options check out the \pai{siunitx} package documentation.

To avoid repeating the options with every use of commands, use the
\begin{lscommand}
  \csi{sisetup}[options: m]
\end{lscommand}
command. It can be used in the preamble to ensure consistent style throughout
the document, or inside the document when the style needs to be changed just
in a given element, such as a table. These options may be also passed as package
options. \autoref{sistandaloneexa} shows an example usage of the package in
a full document.
\begin{listing}
  \begin{example}[standalone, template=empty, examplewidth=\linewidth,paperwidth=12cm, paperheight=5.5cm, vertical_mode]
\documentclass{article}

\usepackage[paperheight=\height,paperwidth=\width,margin=0.3cm]{geometry} %!hide
\sloppy %!hide
\usepackage{booktabs}
\usepackage[per-mode = fraction, unit-color = red]{siunitx}

\begin{document}
Units are typeset like this: \qty{9.81}{\m\per\s\squared}.
\begin{table}[h]
  \centering
  \sisetup{exponent-mode = fixed, fixed-exponent = 1}
  \begin{tabular}{@{}lr@{}}
    \toprule
    No. & Result (\unit{\kg}) \\
    \midrule
    1   & \num{12.3}          \\ 
    2   & \num{0.3}           \\ 
    3   & \num{1e2}           \\
    \bottomrule
  \end{tabular}
  \caption{Measurement results.
    Here custom options are used.}\label{table}
\end{table}

Now we are back to normal: \qty{3.637e-4}{\m\squared\per\s}.
\end{document}
\end{example}
  \caption{An example of using \pai{siunitx} in a document.}%
  \label{sistandaloneexa}
\end{listing}

\subsection{Other \pai{siunitx} commands}

Besides the basic commands described above \pai{siunitx} also defines several
other useful commands.

Angles are special in that they use base $60$ instead of base ten when
subdividing. The command
\begin{lscommand}
  \csi{ang}[options: o, angle: m]
\end{lscommand}
exists in order to simplify entering them. If they are entered using semicolon
as a separator, then the following numbers are treated as minutes and seconds.
If they are given as a decimal number, they will be displayed as such.
\begin{example}
\ang{10}
\qquad %!hide
\ang{1;2;3}
\qquad %!hide
\ang{;;20} \\
\ang{32.5}
\qquad %!hide
\qty{60}{\degree}
\end{example}

When listing several numbers we may use the
\begin{lscommand}
  \csi{numlist}[options: o, numbers: m]
\end{lscommand}
where \carg{numbers} are delimited by semicolons. This command is context aware
which means it will produce correct results when used with \pai{polyglossia}.
\begin{example}
\numlist{1;2;3;4}

\begin{german}
  \numlist{1;2;3;4}
\end{german}
\end{example}

The package contains also dedicated commands for dealing with products and
ranges of numbers. The numbers in product should be delimited with the \verb|x|
letter.
\begin{example}
Pick a number from
range \numrange{1}{10}. \\
\numproduct{2x5} is $10$. \\
\begin{german}
  \numrange{1}{10}
\end{german}
\end{example}

All these commands also have quantity versions.
\begin{example}
Obtained results:
\qtylist{2;5;7}{\L}.\\
Acceptable range is
\qtyrange{1}{10}{\kg}. \\
This area is
\qtyproduct{2x5}{\metre}.
\end{example}

If we are dealing with quantum mechanics then complex numbers may be useful.
To typeset them use \csi{complexnum} and \csi{complexqty}. They can be entered in
both Cartesian form and polar coordinates.

\begin{example}[examplewidth=0.5\linewidth]
The conjugate of
\complexnum{2+3i} is
\complexnum{2-3i}
which is approximately
\complexnum{3.6056:-56.310}. \\
Why is my scale showing my
weight is
\complexqty{65+i21}{\kg}?
\end{example}

Sometimes the required unit that we want to use is missing. For example
throughout this book the unit of typographical point is used when talking about
font sizes. To define it we may use the
\begin{lscommand}
  \csi{DeclareSIUnit}[options:o, unit:m, symbol:m]
\end{lscommand}
command. The \carg{unit} is the macro we will use in the interpreted mode,
while the \carg{symbol} is the typeset symbol. The \carg{options} allows us to
define default options concerning, for example, the spacing of the unit in
quantities.
\begin{example}
\DeclareSIUnit{\pt}{pt}
The default font size
in \LaTeX{} is \qty{10}{\pt}.
\end{example}

This command may also be used when we want to adjust the appearance of
predefined unit. For example, the default litre is typeset using uppercase
\enquote*{L} and you may want to change it to use lowercase \enquote*{l}
instead.
\begin{example}
Litre before redefining:
\unit{\L}. \\
\DeclareSIUnit{\litre}{l}
Litre after redefining:
\unit{\L}.
\end{example}

Similarly the commands
\begin{lscommand}
  \csi{DeclareSIPrefix}    \\
  \csi{DeclareSIPower}     \\
  \csi{DeclareSIQualifier}
\end{lscommand}
allow definition of unit-related macros in case additional are needed.
\begin{example}[examplewidth=0.8\linewidth, vertical_mode]
\DeclareSIUnit{\pt}{pt}
\DeclareSIPower\quartic\tothefourth{4}
\DeclareSIPrefix\decakilo{dk}{4}
\DeclareSIQualifier\polymer{pol}

It's over \qty{9000}{\quartic\decakilo\pt\polymer}!
\end{example}

\subsection{Table columns with numbers}\label{sec:sitables}\index{decimal
  alignment}

When typesetting tables which contain numeric data it is often useful to align
them along the decimal point, so they can be compared easily. The \pai{siunitx}
package adds the special column specifier \cargv{S} for this purpose to the
\ei{tabular} environment. Non numeric data must be surrounded by curly brackets
in such columns.
\begin{listing}
  \begin{example}[examplewidth=0.7\linewidth, vertical_mode]
\begin{tabular}{@{}lS@{}}
  \toprule
  Day       & {Candy eaten (\unit{\g})} \\
  \midrule
  Monday    & .3011                     \\
  Tuesday   & 54.86                     \\
  Wednesday & 1000.9722                 \\
  Thursday  & -1000.9722                \\
  \bottomrule
\end{tabular}
\end{example}
  \caption{A simple example of using \pai{siunitx}'s \cargv{S} column
    specification.}
\end{listing}

When presenting numeric data, such as above, it is good to remember the
following two rules
\begin{itemize}
  \item Never drop the leading zero before the decimal point.
  \item If the unit is the same in all cells put it in the heading.
\end{itemize}
All the numbers in \cargv{S} columns are automatically parsed by \pai{siunitx},
which means that if we use it, the first rule will be enforced for us.

In order to influence parsing\slash{}displaying options of a
single column, pass them in square brackets to the column specification. See
\autoref{lst:siunitxS} for an example.
\begin{listing}
  \begin{example}[examplewidth=0.8\linewidth, vertical_mode]
\DeclareSIUnit{\eur}{\euro}
\begin{tabular} {
    @{}l
    S[output-decimal-marker={,},
      per-mode=symbol]@{}
  }
  \toprule
  Candy        & {Price (\unit{\eur\per\kg})} \\
  \midrule
  Chocolate    & 11.30                        \\
  Lollipops    & 15.86                        \\
  Marshmallows & 5.97                         \\
  Golden taffy & 1125.12                      \\
  \bottomrule
\end{tabular}
\end{example}
  \caption{An example of using optional parameters in a single \cargv{S}
    column specification.}\label{lst:siunitxS}
\end{listing}

Besides standard options there are also table specific options that can be
passed to the columns. An important one is \cargv{table-format}. By default the
\cargv{S} specifier centres the decimal point leaving equal space to the left
and to the right of it. This leads to a lot of empty space if lengths of
integer and fraction parts are widely different as can be seen in
\autoref{lst:toowide}.
\begin{listing}
  \begin{example}[examplewidth=0.7\linewidth, vertical_mode]
\sisetup{exponent-mode = fixed}
\begin{tabular} {@{}SS@{}}
  \toprule
  {\unit{\ug}} & {\unit{\kg}} \\
  \midrule
  1 & 1e-9 \\
  0.2 & 2e-10 \\
  35 & 35e-8 \\
  -100 & -1e-7 \\
  \bottomrule
\end{tabular}
\end{example}
  \caption{An anti-example of using \pai{siunitx}'s \cargv{S} column
    specification without setting the \cargv{table-format}.}\label{lst:toowide}
\end{listing}
We can then use
\cargv{table-format} to instruct \LaTeX{} how much space should be reserved
before and after dot. For example \cargv{6.2} means that it should reserve
space for 6 digits before the decimal point, and for two after. The format may
also contain information about signs, exponents or text around numbers. See
\autoref{lst:tableformat} for an example.
\begin{listing}
  \begin{example}[examplewidth=0.7\linewidth, vertical_mode]
\sisetup{exponent-mode = fixed}
\begin{tabular} {
    @{}
    S[table-format = +3.1]
    S[table-format = +1.10]
    @{}
  }
  \toprule
  \multicolumn{2}{c}{Weight} \\
  \midrule
  {\unit{\ug}} & {\unit{\kg}} \\
  \midrule
  1 & 1e-9 \\
  0.2 & 2e-10 \\
  35 & 35e-8 \\
  -100 & -1e-7 \\
  \bottomrule
\end{tabular}
\end{example}
  \caption{An example of using \pai{siunitx}'s \cargv{S} column
    specification with the \cargv{table-format} key.}\label{lst:tableformat}
\end{listing}

By default numbers and text within the \cargv{S} columns and centred. If you
want to change the alignment use the \cargv{table-number-alignment} and
\cargv{table-text-alignment} options. See \autoref{lst:salignment} for an example.
\begin{listing}
  \begin{example}[examplewidth=0.7\linewidth, vertical_mode]
\begin{tabular} {
    @{}
    S[
      table-format = 2.1(1.1),
      table-number-alignment = left,
      uncertainty-mode = separate,
    ]
    S[
      table-format = 2.2e2{ big!},
      table-text-alignment = right,
    ]
    @{}
  }
  \toprule
  {Confidence (\unit{\percent})}
    & {Value} \\
  \midrule
  57.1(2) & 10e2 \\
  25.3(0) & 2e-1 \\
  0(0.1)  & 99.99e99{ big!} \\
  92(1)      & 7.2e10 \\
  \bottomrule
\end{tabular}
\end{example}
  \caption{An example of aligning text and numbers inside the \pai{siunitx}'s \cargv{S} column.}\label{lst:salignment}
\end{listing}
The numbers may also be aligned inside a \csi{multicol} or \csi{multirow},
using the \csi{tablenum} command as seen in \autoref{lst:tablenum}.
\begin{listing}
  \begin{example}[examplewidth=0.7\linewidth, vertical_mode]
\sisetup{table-format = 4.4e1}
\begin{tabular}{@{}lr@{}}
  \toprule
  Candy     & Friend                         \\
  \midrule
  Chocolate & Peter                          \\
  Lollipop  & Jane                           \\
  \multicolumn{2}{c}{\tablenum{12,34 e0}}    \\
  \multicolumn{2}{c}{\tablenum{333.5567 e1}} \\
  \multicolumn{2}{c}{\tablenum{4563.21 e2}}  \\
  \bottomrule
\end{tabular}
\end{example}
  \caption{An example of using the \csi{tablenum} command.}\label{lst:tablenum}
\end{listing}

If you don't want the decimal alignment but still would like to have
numbers in columns formatted by \pai{siunitx}, change the
\cargv{table-alignment-mode} to \cargv{none}.
\begin{example}[examplewidth=0.45\linewidth]
\sisetup{
  table-alignment-mode = none,
  output-decimal-marker = {,},
  unit-color = red,
}
\begin{tabular}{@{}cS@{}}
  \toprule
  Variable & {Value} \\
  \midrule
  foo & 101.892 \\
  bar & 2e-1 \\
  baz & \qty{2.1}{\kg}  \\
  \bottomrule
\end{tabular}
\end{example}

\section{Arrays and Matrices}%
\label{sec:arraymat}

To typeset \emph{arrays}, use the \ei{array} environment. It works
in a similar way to the \ei{tabular} environment. The \csi{\bs} command is
used to break the lines:
\begin{example}
  \begin{equation*}
    \mathbf{X} = \left( 
      \begin{array}{ccc}
        x_1 & x_2 & \ldots \\
        x_3 & x_4 & \ldots \\
        \vdots & \vdots & \ddots
      \end{array} \right)
  \end{equation*}
\end{example}

The \ei{array} environment can also be used to typeset \wi{piecewise function}s by
using a ``\cargv{.}'' as an invisible \csi{right} delimiter:
\begin{example}
\begin{equation*}
  |x| = \left\{
    \begin{array}{rl}
      -x & \text{if } x < 0,\\
      0 & \text{if } x = 0,\\
      x & \text{if } x > 0.
    \end{array} \right.
\end{equation*}
\end{example}
The \ei{cases} environment from \pai{amsmath} simplifies
the syntax, so it is worth a look:
\begin{example}
  \begin{equation*}
    |x| = 
    \begin{cases}
      -x & \text{if } x < 0,\\
      0 & \text{if } x = 0,\\
      x & \text{if } x > 0.
    \end{cases} 
\end{equation*}
\end{example}

Matrices\index{matrix} can be typeset by \ei{array}, but
\pai{amsmath} provides a better solution using the different \ei{matrix}
environments. There are six versions with different delimiters: \ei{matrix}
(none), \ei{pmatrix} $($, \ei{bmatrix} $[$, \ei{Bmatrix} $\{$, \ei{vmatrix} $\vert$ and
\ei{Vmatrix} $\Vert$. You don't have to specify the number of columns as with
\ei{array}. The maximum number is 10, but it is customisable (though it is not
very often you need 10 columns!):
\begin{example}
\begin{equation*}
  \begin{matrix} 
    1 & 2 \\
    3 & 4 
  \end{matrix} \qquad
  \begin{bmatrix} 
    p_{11} & p_{12} & \ldots 
    & p_{1n} \\
    p_{21} & p_{22} & \ldots 
    & p_{2n} \\
    \vdots & \vdots & \ddots 
    & \vdots \\
    p_{m1} & p_{m2} & \ldots 
    & p_{mn} 
  \end{bmatrix}
\end{equation*}
\end{example}

\section{Spacing in Math Mode}\label{sec:math-spacing}%
\index{math spacing}
If the spacing within formulae chosen by \LaTeX{}
is not satisfactory, it can be adjusted by inserting special spacing
commands: \csi{,} for $\frac{3}{18}\:\textrm{quad}$
(\demowidth{0.166em}), \csi{:} for $\frac{4}{18}\: \textrm{quad}$
(\demowidth{0.222em}) and \csi{;} for $\frac{5}{18}\: \textrm{quad}$
(\demowidth{0.277em}).  The escaped space character \csi{\textvisiblespace}
generates a medium sized space comparable to the interword spacing and
\csi{quad} (\demowidth{1em}) and \csi{qquad} (\demowidth{2em}) produce
large spaces. The size of a \csi{quad} corresponds to the width of the
character `M' of the current font. \csi{!} produces a
negative space of $-\frac{3}{18}\:\textrm{quad}$
($-$\demowidth{0.166em}).

\begin{example}
\begin{equation*}
  \int_1^2 \ln x \mathrm{d}x 
  \qquad
  \int_1^2 \ln x \,\mathrm{d}x
\end{equation*}
\end{example}

Note that `d' in the differential is conventionally set in roman.
In the next example, we define a new command \csi{ud} (upright d) which produces
``$\,\mathrm{d}$'' (notice the spacing \demowidth{0.166em} before the
$\text{d}$), so we don't have to write it every time. The \csi{NewDocumentCommand} is
placed in the preamble.  More on
\csi{NewDocumentCommand} in \autoref{sec:new_commands} on \autopageref{sec:new_commands}.
\begin{example}[examplewidth=0.35\linewidth]
\NewDocumentCommand{\ud}{}{\,\mathrm{d}}

\begin{equation*}
 \int_a^b f(x)\ud x 
\end{equation*}
\end{example}

If you want to typeset multiple integrals, you'll discover that the spacing
between the integrals is too wide. You can correct it using \csi{!}, but
\pai{amsmath} provides an easier way for fine-tuning
the spacing, namely the \csi{iint}, \csi{iiint}, \csi{iiiint}, and \csi{idotsint}
commands.

\begin{example}[examplewidth=0.35\linewidth]
\NewDocumentCommand{\ud}{}{\,\mathrm{d}}

\begin{IEEEeqnarray*}{c}
  \int\int f(x)g(y) 
                  \ud x \ud y \\
  \int\!\!\!\int 
         f(x)g(y) \ud x \ud y \\
  \iint f(x)g(y)  \ud x \ud y 
\end{IEEEeqnarray*}
\end{example}

For further details see~\cite{amstestmath} or Chapter~8 of \companion{}.

\subsection{Phantoms}

When vertically aligning text using \ai{\^} and \ai{\_} \LaTeX{} is sometimes
just a little too helpful. Using the \csi{phantom} command you can
reserve space for characters that do not show up in the final output.
The easiest way to understand this is to look at an example:
\begin{example}
\begin{equation*}
{}^{14}_{6}\text{C}
\qquad \text{versus} \qquad
{}^{14}_{\phantom{1}6}\text{C}
\end{equation*}
\end{example}
If you want to typeset a lot of isotopes as in the example, the \pai{mhchem}
package is very useful for typesetting isotopes and chemical formulae too.

\section{Fiddling with the Math Fonts}\label{sec:fontsz}
Different math fonts are listed on \autoref{mathalpha} on \autopageref{mathalpha}.
\begin{example}
 $\Re \qquad
  \mathcal{R} \qquad
  \mathfrak{R} \qquad
  \mathbb{R} \qquad $  
\end{example}
The last two require \pai{amssymb} or \pai{amsfonts}.

Sometimes you need to tell \LaTeX{} the correct font
size. In math mode, this is set with the following four commands:
\begin{lscommand}
  \csi{displaystyle}~($\displaystyle 123$)\\
  \csi{textstyle}~($\textstyle 123$)\\
  \csi{scriptstyle}~($\scriptstyle 123$)\\
  \csi{scriptscriptstyle}~($\scriptscriptstyle 123$)
\end{lscommand}

If $\sum$ is placed in a fraction, it'll be typeset in text style unless you tell
\LaTeX{} otherwise:
\begin{example}[examplewidth=0.55\linewidth]
\begin{equation*}
 P = \frac{\displaystyle{ 
   \sum_{i=1}^n (x_i- x)
   (y_i- y)}} 
   {\displaystyle{\left[
   \sum_{i=1}^n(x_i-x)^2
   \sum_{i=1}^n(y_i- y)^2
   \right]^{1/2}}}
\end{equation*}    
\end{example}
Changing styles generally affects the way big operators and limits are displayed.

% This is not a math accent, and no maths book would be set this way.
% mathop gets the spacing right.

\subsection{Bold Symbols}%
\index{bold symbols}

It is quite difficult to get bold symbols in \LaTeX{}; this is
probably intentional as amateur typesetters tend to overuse them.  The
font change command \csi{mathbf} gives bold letters, but these are
roman (upright) whereas mathematical symbols are normally italic, and
furthermore it doesn't work on lower case Greek letters.
There is a \csi{boldmath} command, but \emph{this can only be used
  outside math mode}. It works for symbols too, though:
\begin{example}
$\mu, M \qquad 
\mathbf{\mu}, \mathbf{M}$
\qquad \boldmath{$\mu, M$}
\end{example}

The package \pai{amsbsy} (included by \pai{amsmath}) as well as the
package \pai{bm} from the \pai{tools} bundle make this much easier as they include
a \csi{boldsymbol} command:

\begin{example}
$\mu, M$
$\boldsymbol{\mu}
  \boldsymbol{M}$
\end{example}

\section{Theorems, Lemmas, \ldots}

When writing mathematical documents, you probably need a way to
typeset ``Lemmas'', ``Definitions'', ``Axioms'' and similar
structures.
\begin{lscommand}
  \csi{newtheorem}[name:m, counter:o, text:m, section:o]
\end{lscommand}
The \carg{name} argument is a short keyword used to identify the
``theorem''. With the \carg{text} argument you define the actual name
of the ``theorem'', which will be printed in the final document.

The arguments in square brackets are optional. They are both used to
specify the numbering used on the ``theorem''. Use  the \carg{counter}
argument to specify the \carg{name} of a previously declared
``theorem''. The new ``theorem'' will then be numbered in the same
sequence.  The \carg{section} argument allows you to specify the
sectional unit within which the ``theorem'' should get its numbers.

After executing the \csi{newtheorem} command in the preamble of your
document, you can use the following command within the document.
\begin{code}
\verb|\begin{|\emph{name}\verb|}[|\emph{text}\verb|]|\\
This is my interesting theorem\\
\verb|\end{|\emph{name}\verb|}|     
\end{code}

The \pai{amsthm} package (part of \hologo{AmSLaTeX}) provides the
\begin{lscommand}
  \csi{theoremstyle}[style: m]
\end{lscommand}
command which lets you define what the theorem is all about by picking
from three predefined styles: \cargv{definition} (fat title, roman body),
\cargv{plain} (fat title, italic body) or \cargv{remark} (italic
title, roman body).

This should be enough theory. The following examples should
remove any remaining doubt, and make it clear that the
\csi{newtheorem} command is way too complex to understand.

First define the theorems:

\begin{minted}{latex}
\theoremstyle{definition} \newtheorem{law}{Law}
\theoremstyle{plain}      \newtheorem{jury}[law]{Jury}
\theoremstyle{remark}     \newtheorem*{marg}{Margaret}
\end{minted}

\begin{example}
\begin{law}\label{law:box}
Don't hide in the witness box
\end{law}
\begin{jury}[The Twelve]
It could be you! So beware and
see law~\ref{law:box}.\end{jury}
\begin{jury}
You will disregard the last
statement.\end{jury}
\begin{marg}No, No, No\end{marg}
\begin{marg}Denis!\end{marg}
\end{example}

The ``Jury'' theorem uses the same counter as the ``Law''
theorem, so it gets a number that is in sequence with
the other ``Laws''. The argument in square brackets is used to specify
a title or something similar for the theorem.
\begin{example}[examplewidth=0.45\linewidth]
%!showbegin !hide
\newtheorem{mur}{Murphy}[section]
%!showend !hide

\begin{mur} If there are two or 
more ways to do something, and 
one of those ways can result in
a catastrophe, then someone 
will do it.\end{mur}
\end{example}

The ``Murphy'' theorem gets a number that is linked to the number of
the current section. You could also use another unit, for example chapter or
subsection.

If you want to customise your theorems down to the last dot, the
\pai{ntheorem} package offers a plethora of options.

\subsection{Proofs and End-of-Proof Symbol}\label{sec:putting-qed-right}

The \pai{amsthm} package also provides the \ei{proof} environment.

\begin{example}
\begin{proof}
 Trivial, use
 \begin{equation*}
   E=mc^2.
 \end{equation*}
\end{proof}
\end{example}

With the command \csi{qedhere} you can move the `end of proof' symbol
around for situations where it would end up alone on a line.

\begin{example}
\begin{proof}
 Trivial, use
 \begin{equation*}
   E=mc^2. \qedhere
 \end{equation*}
\end{proof}
\end{example}

Unfortunately, this correction does not work for \ei{IEEEeqnarray}:
\begin{example}
\begin{proof}
  This is a proof that ends
  with an equation array:
  \begin{IEEEeqnarray*}{rCl}
    a & = & b + c \\
    & = & d + e. \qedhere
  \end{IEEEeqnarray*}  
\end{proof}
\end{example}
\noindent
The reason for this is the internal structure of \ei{IEEEeqnarray}:
it always puts two invisible columns at both sides of the array that
only contain a stretchable space. By this \ei{IEEEeqnarray} ensures
that the equation array is horizontally centred. The
\csi{qedhere} command should actually be put \emph{outside} this
stretchable space, but this does not happen as these columns are
invisible to the user.

There is a very simple remedy. Define the stretching
explicitly!
\begin{example}
\begin{proof}
  This is a proof that ends
  with an equation array:
  \begin{IEEEeqnarray*}{+rCl+x*}
    a & = & b + c \\
    & = & d + e. & \qedhere
  \end{IEEEeqnarray*}  
\end{proof}
\end{example}
\noindent
Note that the \cargv{+} in \verb|{+rCl+x*}| denotes stretchable spaces, one
on the left of the equations (which, if not specified, will be done
automatically by \ei{IEEEeqnarray}!) and one on the right of the
equations. But now on the right, \emph{after} the stretching column,
we add an empty column \cargv{x}. This column will only be needed on
the last line if the \csi{qedhere} command is put
there. Finally, we specify a \cargv{*}. This is a null-space that
prevents \ei{IEEEeqnarray} from adding another unwanted \cargv{+}-space!

In the case of equation numbering, there is a similar problem. Comparing
\begin{example}
\begin{proof}
  This is a proof that ends
  with a numbered equation:
  \begin{equation}
    a = b + c.
  \end{equation}
\end{proof}
\end{example}
\noindent
with
\begin{example}
\begin{proof}
  This is a proof that ends
  with a numbered equation:
  \begin{equation}
    a = b + c. \qedhere
  \end{equation}
\end{proof}
\end{example}
\noindent
you notice that in the (correct) second version the $\Box$ is much
closer to the equation than in the first version.

Similarly, the correct way of putting the QED-symbol at the end of an
equation array is as follows:
\begin{example}
\begin{proof}
  This is a proof that ends
  with an equation array:
  \begin{IEEEeqnarray}{+rCl+x*}
    a & = & b + c \\
    & = & d + e. \\
    &&& \qedhere\nonumber
  \end{IEEEeqnarray}  
\end{proof}
\end{example}
\noindent
which contrasts with
\begin{example}
\begin{proof}
  This is a proof that ends
  with an equation array:
  \begin{IEEEeqnarray}{rCl}
    a & = & b + c \\
    & = & d + e.
  \end{IEEEeqnarray}  
\end{proof}
\end{example}

%

% Local Variables:
% TeX-master: "lshort"
% mode: latex
% mode: flyspell
% End:
